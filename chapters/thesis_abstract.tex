\begin{abstract}

Neutrino mixing and oscillations have been well studied during the last few decades.
Nevertheless, several unanswered questions remain, in particular CP violation and the ordering of the neutrino masses.
\dune{}, a next-generation long-baseline neutrino oscillation experiment is being built to answer them.
Reaching the sensitivity goal will require a data sample of unprecedented size and precision.
Therefore, a high-intensity beam and a massive detector providing excellent tracking and calorimetry are required.

A Liquid Argon Time Projection Chamber (\lartpc{}) fulfils this requirements very well and was therefore chosen for the far detector.
To bring systematic uncertainties down to an acceptable level, a \lartpc{} near detector component is required.
The high-multiplicity \dune{} near detector environment poses significant challenges.

\lartpc{}s have used projective wire readouts for charge detection since their conception in 1977.
However, a wire readout introduces intrinsic ambiguities in event reconstruction which is problematic in high-multiplicity environments.
To overcome these limitations, a pixelated charge readout for \lartpc{} was developed and successfully tested.
A software framework was developed to reconstruct the recorded cosmic muon tracks employing a Kalman filter.
Pixelated charge readout systems represent the single largest advancement in the sensitivity of LArTPCs, enabling true 3D tracking.
They are mechanically robust and the direct 3D readout minimises reconstruction ambiguities, reducing event pile-up and improving background rejection.

Due to their increased number of channels, pixelated charge readouts give raise to the need for novel readout electronics.
Existing wire readout electronics were tested and found unsuitable for the new readout.
Together with the successful demonstration of the readout itself, this triggered the development of bespoke pixel electronics for the \dune{} near detector.

The large volumes required by the \dune{} \lartpc{}s result in longer drift distances and thus require higher drift voltages.
Recent studies have shown the dielectric strength of \lar{} to be much lower than predicted by earlier work.
This triggered an in-depth study of electric breakdowns including the recording of high-speed footage, current-voltage characteristics, and spectrometry.
As a result of these studies, a method was developed to increase the dielectric strength of \lar{} by an order of magnitude.
However, this technique is not suitable for long-running physics experiments.
Instead, it was found, that it is required to keep fields below \SI{40}{\kilo\volt\per\centi\metre} at all points in the detector to guarantee a safe operation.

The entirety of the R\&D effort lead to the development of a new fully-modular, pixelated \lartpc{} concept---ArgonCube.
It is aimed to address the studied challenges.
Splitting the detector volume into small self-contained TPCs sharing a common \lar{} bath, reduces the required drift voltages to a handleable level.
A pixelated charge readout paired with bespoke electronics will exploit true 3D tracking to cope with the expected event pile-up.

The work is topped off with a study of the near detector event pile-up.
Simulated \Pgpz decays were reconstructed by a rudimentary cone-based algorithm assuming unambiguous 3D information on charge deposition with a resolution of \SI{3}{\milli\metre}.
It was found that the impact of pile-up on the reconstructed neutrino energy is in the range of \SIrange{2}{3}{\percent} on average and below \SI{0.1}{\percent} for $> \SI{50}{\percent}$ of the neutrino events.

At the time of this writing, \AC{} is the top candidate for the \lar{} component of the \dune{} near detector.

\end{abstract}

\clearpage