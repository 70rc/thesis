\chapter{Feasibility Study of a Pixelated \lartpc{} for the \dune{} Near Detector}
\label{chap:dune-nd}

With \AC{} selected as  the \lar{} component of the \dune{} near detector (ND), there were two main question that needed to be addressed:
\begin{enumerate}
	\item Is a pixelated \lartpc{} feasible?
	\item Can the \lar{} detector handle the high rates?
\end{enumerate}
Number one is addressed in Chapter~\ref{chap:viper}.
This chapter will address question number two.
More details on \dune{} can be found in Section~\ref{sec:nu-detection_dune} while the proposed \AC{} \lar{} component of the ND is described in Chapter~\ref{chap:argoncube}.

\begin{figure}[htb]
	\centering
	\includegraphics[width=\textwidth]{pile-up/charge_flux}
	\caption{Average current collected by the readout for one spill as a function of time.
	The current is given in arbitrary but equal units for both plots.
	The upper plot assumes the whole charge is deposited instantaneously while for the lower plot, the actual spill duration from~\cite{dune2} is used.}
	\label{fig:dune-nd_charge-flux}
\end{figure}

\lartpc{}s are intrisically slow detectors with a readout time of $\approx \SI{0.5}{\milli\second\per\metre}$ drift length for a \SI{1}{\kilo\volt\per\centi\metre} drift field (see Chapter~\ref{chap:lartpc}).
This causes a pile-up of events in the detector; if the latter was infinitely fast, all neutrino interactions could be separated in time.
In reality, even the \AC{} TPCs with a drift length of only \SI{0.5}{\metre}, corresponding to a full readout cycle of \SI{250}{\micro\second}, are significantly slower than the spill duration of \SI{10}{\micro\second} of the \dune{} beamline reference design.~\cite{dune2}
Figure~\ref{fig:dune-nd_charge-flux} visualises this effect.
The charge arriving at the readout is represented as an average current in arbitrary units (the same for top and bottom, though).
The magnitude of this readout current is a direct measure for event pile-up in the corresponding time slice.
For simplicity, an infinitely short spill duration was assumed for the pile-up study (top), i.e.\ the whole ionisation charge produced by one beam spill is deposited instantaneously inside the TPC volume.
As the time in between beam spills is $\order{\SI{1}{\second}}$, all this charge can be read out within one drift time.
In this case, the average current (pile-up) seen by the readout is constant over the whole readout cycle.
The realistic case with the spill duration of the reference beam is depicted in the bottom plot.
At the beginning of the readout cycle, there is no charge deposited yet, the current (pile-up) is zero.
Over the duration of the beam spill, ionisation charge accumulates inside the TPC volume while the exisiting charge is transported towards the readout by the drift field.
After the beam spill is over, the remainder of the initial drift volume (\SI{240}{\micro\second}) contains a uniform charge density.
The additional \SI{10}{\micro\second}, in front of the cathode and part of the next readout cycle, again, contain a ramped charge density with zero charge at the cathode, corresponding to the end of the spill.
In short, a spill duration shorter than but comparable to the drift time results in the shape of the ionisation current (event pile-up) seen over time to become a trapecoid rather than a square.
The integral, i.e.\ the total ionisation charge (deposited energy) is the same but part of it is shifted from the spill time slice to the beginning of the next readout cycle.
Similarly, the peak current (pile-up) is the same, as long as the spill duration is shorter than the drift time.
If the spill duration becomes longer than the drift time, the charge is distributed over more then two readout cycles and the peak current (pile-up) begins to decrease.
Therefore, the assumption of an infinitely short spill is a worst-case scenario slightly improved by the real, finite spill duration.
However, for most of the drift time (\SI{240}{\micro\second}), pile-up is the same.


\section{The Argon Box Simulation Tool}
\label{sec:dune-nd_argon-box}

To simulate the expected neutrino interactions in the ND, the Argon Box\footnote{\url{https://github.com/dadwyer/argon_box}} simulation tool was used.
The neutrino group at LBNL is developing it with the goal of providing an easy-to-use simulation of particle interactions in \lar{} component of the ND.
Primary particles can either be provided by a particle gun (e.g.\ \Pem, \Pn, \Pp, \Pgmp) or in form of a \emph{HEPEVT} file\footnote{A file format standard for passage of particle events between different simulation tools}.
For this study, \num{6.6e6} neutrino events were produced with the GENIE\footnote{\url{https://genie.hepforge.org}} neutrino event generator.
Secondary particle transport and interaction in Argon Box is performed by Geant4.\footnote{\url{http://geant4.cern.ch}}
Finally, the energy deposition in \lar{} is voxelised and stored together with all the necessary ancillary information about the depositing particle.
The data is stored in the tree format of the ROOT data analysis framework\footnote{\url{https://root.cern}}.
This allows for convenient further processing using ROOT.

\afterpage{\clearpage}


\section{\Pgpz pile-up study}
\label{sec:dune-nd_pile-up}

A significant amount of \Pgpz are produced in several resonant and coherent neutrino interactions (see Table~\ref{tab:nu-detection_nd-rates} Section~\ref{sec:nu-detection_interactions} and~\cite{dune2}).
They decay according to
\begin{IEEEeqnarray}{C}
	\HepProcess{\Pgpz \to \Pgg\Pgg}
\end{IEEEeqnarray}
with a branching ratio of \SI{98.8}{\percent}\cite{pdg}.
The photons subsequently produce electromagnetic showers in \lar{} (see Section~\ref{sec:nu-detection_fs}).
At the energies of the \dune{} beam (see Figure~\ref{fig:nu-detection_dune-flux}), most showers do not deposit a homogenous cone of charge but rather a lot of individually resolvable \Pepm tracks.
More importantly, there often are significatn gaps in betweend these charge clusters.
A main challenge of shower reconstruction is to associate these well separated charge blobs to the correct event.
Misidentifications lead to a misreconstruction of the neutrino energy.
The resulting discrepancy to the true neutrino energy has the potential to skew the measured energy spectrum and, thus, influence the oscillation measurements.
The complexity of reconstruction paired with the potential impact on the physics measurements makes photons produced by \Pgpz decays a good sample to study the robustness to pile-up of a pixelated \lartpc{} in the \dune{} ND environment.

\begin{table}[htb]
	\centering
	\caption{Parameters of the \Pgpz pile-up simulation.}
	\label{tab:dune-nd_pile-up-params}
	\begin{tabu} to \textwidth {|l|S|}
		\hline
		{$X$-axis} &					{Drift} \\
		\hline
		{$Y$-axis} &					{Vertical} \\
		\hline
		{$Z$-axis} &					{Beam} \\
		\hline
		{Resolution $X$} &				\SI{0.3}{\centi\metre} \\
		\hline
		{Resolution $Y$} &				\SI{0.3}{\centi\metre} \\
		\hline
		{Resolution $Z$} &				\SI{0.3}{\centi\metre} \\
		\hline
		{Target volume $X$} &			\SIrange{-100}{500}{\centi\metre} \\
		\hline
		{Active volume $X$} &			\SIrange{0}{400}{\centi\metre} \\
		\hline
		{Fiducial volume $X$} &			\SIrange{30}{370}{\centi\metre} \\
		\hline
		{Target volume $Y$} &			\SIrange{-100}{350}{\centi\metre} \\
		\hline
		{Active volume $Y$} &			\SIrange{0}{250}{\centi\metre} \\
		\hline
		{Fiducial volume $Y$} &			\SIrange{30}{220}{\centi\metre} \\
		\hline
		{Target volume $Z$} &			\SIrange{-400}{500}{\centi\metre} \\
		\hline
		{Active volume $Z$} &			\SIrange{0}{500}{\centi\metre} \\
		\hline
		{Fiducial volume $Z$} &			\SIrange{30}{470}{\centi\metre} \\
		\hline
		{Detection threshold} &			\SI{0.1}{\mega\electronvolt} \\
		\hline
		{Cone extent} &					{$10 X_{\m{0}} = \SI{140}{\centi\metre}$} \\
		\hline
		{Cone aperture (full angle)} &	\ang{30} \\
		\hline
		{Cylinder diameter} &			\SI{5}{\centi\metre} \\
		\hline
		{Beam intensity} &				{$\SI{2}{\mega\watt} \widehat{=} \SI{0.2}{evt\per\tonne}$} \\
		\hline
	\end{tabu}
\end{table}

To investigate the effects of pile-up on energy reconstruction, a working reconstruction algorithm is necessary.
However, at the time of writing, official reconstruction tools were only available for \lartpc{}s read out by wire planes.\footnote{\url{http://larsoft.org}}
Therefore, a rather primitive algorithm for true 3D space points was implemented, under the assumption that a positive outcome of such a pile-up study would imply an even better performance of a more sophisticated reconstruction.
This algorithm is explained in the following, its parameters are listed in Table~\ref{tab:dune-nd_pile-up-params}.

The basic underlying assumption is that a pixel readout without analogue multiplexing will yield unambiguous 3D space points of charge deposition with a given resolution, depending on the geometry of the pixel plane, time resolution of the readout electronics, and charge transport effects.
Chapter~\ref{chap:viper} proves that this is feasible provided the current reconstruction ambiguities can eliminated by a successful deployment of the \larpix{} charge readout electronics described in Section~\ref{sec:electronics_larpix}.
The spatial resolution of the pixel readout is assumed to be \SI{0.3}{\centi\metre} based on the ND design specified in Section~\ref{sec:ac_dune-nd}.
A conservative value of \SI{0.3}{\centi\meter} was chosen in drift direction.
This has several advantages.
Choosing the same resolution as the pixel pitch makes the simulation independent of the orientation of the TPC.
MicroBooNE has achieved a resolution in drift direction $< \SI{3}{\milli\metre}$~\cite{uboone}, making it safe to assume \larpix{} will have a similar performance, even though not yet fully characterised.
Finally, a conservative value also accounts for charge diffusion.
Furthermore, it is assumed that EM showers can be identified and their starting point and direction reconstructed with negligible errors and inefficiencies, i.e.\ this information is taken from the simulation truth.
A cone is calculated in the direction of the shower with its tip at the first charge deposition of the initial photon.
The opening angle and longituginal extent of the cone were optimised by looking at the distributions of the distance from the starting point and angle w.r.t. the direction of the shower.
The finite resolution of the detector is emulated by voxelising the charge deposition with the corresponding resolution in all three spatial coordinates.
This leads to problems near the tip of the cone where the transversal extent is lower than the voxel dimensions.
In particular, it can happen that most of the initial charge is shifted outside the cone.
Furthermore, multiple scattering at lower energies makes the cone model suboptimal near the tip.
Therefore, the acceptance volume for the reconstruction is taken as the union of the cone with a cylinder around the direction of the shower of the same longitudinal extent as the cone.
The cylinder radius was tuned to optimise the trade-off between increased efficiency and decreased purity as defined below.

\begin{figure}[htb]
	\centering
	\includegraphics[width=\textwidth]{pile-up/2MW/eff_pur}
	\caption{Efficiency (accepted true over total true energy deposition) and purity (accepted true over total accepted energy deposition) distributions of a simple cone-cylinder union EM shower reconstruction algorithm.
	The distribution represents the fraction of photons produced by \Pgpz whose energy was reconstructed with the corresponding efficiency/purity.
	Purity is shown for four different selections of misidentified energy: total (magenta); deposited by other events only (cyan); deposited by other events only, excluding muons (red), deposited by photons and neutrons from other events only (blue).
	The simulated beam intensity is \SI{2}{\mega\watt} at \SI{80}{\giga\electronvolt} proton energy.}
	\label{fig:dune-nd_2MW-eff-pur}
\end{figure}

Argon Box propagates the neutrino interaction events it gets from GENIE through liquid argon, the output is a ROOT tree of neutrino interaction events.
To get a realistic simulation of beam events in the detector, these events need to be distributed randomly in time and space.
Beam spills are simulated by drawing the number of events for each spill from a Poisson distribution whose mean is calculated from the beam intensity and the target mass according to the values in Table~\ref{tab:nu-detection_beam-params}.
The resulting number of events is taken from the Argon Box ROOT tree and their vertices are placed within the \lar{} volume at coordinates drawn from a uniform distribution.
Combined with the target mass given in Table~\ref{tab:dune-nd_pile-up-params}, this results in an equivalent of $\approx \num{1.5e19}$ protons on target (POT).
The seemingly low number (compared to Table~\ref{tab:nu-detection_nd-rates} for instance) is the result of many neutrino interactions happening outside of the active detector.

\begin{figure}[htb]
	\centering
	\includegraphics[width=\textwidth]{pile-up/2MW/energy_error}
	\caption{Fractional error on the reconstructed neutrino energy due to missed or misidentified energy depositions as a function of true neutrino energy.
	Misidentified energy is shown for three different selections: total deposited by other events (cyan); deposited by other events excluding muons (red), deposited by photons and neutrons from other events only (blue).
	The simulated beam intensity is \SI{2}{\mega\watt} at \SI{80}{\giga\electronvolt} proton energy.}
	\label{fig:dune-nd_2MW-energy-error}
\end{figure}

Three different argon volumes are assumed for the simulation: target, active, and fiducial volume.
The actual detector dimensions are represented by the active volume.
It is inside the target volume which is the volume within which the neutrino vertices are placed randomly.
This is done as a crude emulation of rock events, secondary particles from beam neutrino interactions outside the detector volume.
The additional target mass is \SI{1}{\metre} in all four directions transverse to the beam, and \SI{4}{\metre} in upstream beam direction.
According to Equations~\eqref{eq:nu-detection_hardon-long} and~\eqref{eq:nu-detection_hardon-trans}, hadronic showers up to \SI{10}{\giga\electronvolt} are contained $> \SI{95}{\percent}$ longitudinally and $> \SI{50}{\percent}$ transversally (because Equation~\eqref{eq:nu-detection_hardon-trans} gives the radius for \SI{95}{\percent} containment) in the additional volume.
Or in other words, increasing the target volume further will not result in significantly more rock events entering the active volume.
These numbers are supported by Geant4 simulations~\cite{hardonContChris}.
As mentioned in Section~\ref{sec:nu-detection_fs}, EM interactions happen on smaller scales than hadronic interactions.
The big exception are muons due to their high range.
However, as will be explained below, due to the high reconstruction efficiency, it makes sense to ignore pile-up from muons anyway.
Finally, a fiducial volume \SI{30}{\centi\metre} ($\approx 2 X_{\m{0}}$) smaller than the active volume on all six faces is defined.
Without the latter, there is a significant number of photons produced by \Pgpz decays inside the detector but only showering outside the detector.
This selection results in $\approx \num{5.5e5}$ processed \Pgpz photons from the initial \num{6.6e6} neutrino events.
Table~\ref{tab:dune-nd_pile-up-params} contains a summary of all the \lar{} volume dimensions.

\begin{figure}[htb]
	\centering
	\includegraphics[width=\textwidth]{pile-up/2MW_XZ/eff_pur}
	\caption{Efficiency (accepted true over total true energy deposition) and purity (accepted true over total accepted energy deposition) distributions of a simple cone-cylinder union EM shower reconstruction algorithm.
	The distribution represents the fraction of photons produced by \Pgpz whose energy was reconstructed with the corresponding efficiency/purity.
	Purity is shown for four different selections of misidentified energy: total (magenta); deposited by other events only (cyan); deposited by other events only, excluding muons (red), deposited by photons and neutrons from other events only (blue).
	The simulated beam intensity is \SI{2}{\mega\watt} at \SI{80}{\giga\electronvolt} proton energy.
	As a primitive simulation of a wire readout, only X and Z coordinates are used for the energy reconstruction.}
	\label{fig:dune-nd_2MW-XZ-eff-pur}
\end{figure}

\begin{figure}[htb]
	\centering
	\includegraphics[width=\textwidth]{pile-up/2MW_XZ/energy_error}
	\caption{Fractional error on the reconstructed neutrino energy due to missed or misidentified energy depositions as a function of true neutrino energy.
	Misidentified energy is shown for three different selections: total deposited by other events (cyan); deposited by other events excluding muons (red), deposited by photons and neutrons from other events only (blue).
	The simulated beam intensity is \SI{2}{\mega\watt} at \SI{80}{\giga\electronvolt} proton energy.
	As a primitive simulation of a wire readout, only X and Z coordinates are used for the energy reconstruction.}
	\label{fig:dune-nd_2MW-XZ-energy-error}
\end{figure}

After all events of one spill are placed inside the target volume, all \Pgpz photons produced inside the fiducial volume are reconstructed using the cone algorithm.
All energy depositions inside the active volume are considered.
To assess the performance of the performance of the algorithm, the following quantities are computed for each photon:
\begin{description}
	\item[Missed energy] is the energy deposited by the corresponding \Pgpz photon (or its descendants) that is outside of the cone-cylinder union and therefore ``missed'' by the algorithm.
		This is a measure of the reconstruction performance and can be used to ensure optimum tuning of the union parameters.
	\item[Misidentified energy] is the energy inside the cone-cylinder union deposited by descendants of a different (``wrong'') parent neutrino.
		This is a measure of event pile-up: the higher the charge deposition by other events inside the union, the higher the event pile-up.
\end{description}
Using this general definition of misidentified energy leads to quite mediocre results.
However, there are some assumptions that can be taken even without knowing the actual reconstruction algorithm.
From results of earlier experiments\todo{sauce}, the muon reconstruction can be assumed to be very efficient.
Assuming \SI{100}{\percent} reconstruction efficiency for muons and \SI{0}{\percent} for all other particles can therefore serve as an upper limit for misidentified energy.
It can be calculated by ignoring energy deposited by muons originating from other parent neutrinos.
On the other hand, a lower limit for misidentified energy can be calculated by assuming \SI{100}{\percent} reconstruction efficiency for all charged particles and \SI{0}{\percent} for neutral particles (\Pgpz and \Pgg).
This is calculated by only taking into account misidentified energy deposited by neutral particles.
Even assuming \SI{0}{\percent} reconstruction efficiency for neutral particles is potentially too pessimistic.
Future, more sophisticated reconstruction algorithms (e.g. based on machine learning) might be able to partially reconstruct the topology of charge depositions originating from neutral particles and thus prevent their misidentification.
Therefore, it can be assumed that the actual pile-up-related energy reconstruction error is closer to the lower limit and potentially even below.

To make the interpretation of the results as intuitive as possible, missed and misidentified energy are plotted as a function of true photon and neutrino energy, respectively.
As mentioned above, the missed energy is used to measure the performance of the employed photon reconstruction algorithm.
Therefore, it is sensible to compare it to the true photon energy rather than the true energy of its parent neutrino.
On the other hand, the primary goal of this study is to assess the effect of event pile-up on the neutrino energy spectrum.
The misidentified energy is thus compared to the true neutrino energy.
Additionally, it is illustrative to look at the fraction of events with a certain misidentified (missed) energy.
All the aforementioned information is contained in 2D histograms of all events with the true neutrino (photon) energy on one axis and the misidentified (missed) energy on the other axis.
The energy dependence of the error can be obtained by looking at the true energy axis and calculating the mean misidentified (missed) energy for each bin (a profile of the 2D histogram).
Looking at the misidentified (missed) energy axis and summing over all true energies yields the number (fraction) of events with the corresponding misidentified (missed) energy (a projection of the 2D histogram).

\begin{figure}[htb]
	\centering
	\includegraphics[width=\textwidth]{pile-up/10MW/eff_pur}
	\caption{Efficiency (accepted true over total true energy deposition) and purity (accepted true over total accepted energy deposition) distributions of a simple cone-cylinder union EM shower reconstruction algorithm.
	The distribution represents the fraction of photons produced by \Pgpz whose energy was reconstructed with the corresponding efficiency/purity.
	Purity is shown for four different selections of misidentified energy: total (magenta); deposited by other events only (cyan); deposited by other events only, excluding muons (red), deposited by photons and neutrons from other events only (blue).
	The simulated beam intensity is \SI{10}{\mega\watt} at \SI{80}{\giga\electronvolt} proton energy.}
	\label{fig:dune-nd_10MW-eff-pur}
\end{figure}

The results of the pile-up study are shown in Figures~\ref{fig:dune-nd_2MW-eff-pur} through~\ref{fig:dune-nd_10MW-energy-error}.
Figures~\ref{fig:dune-nd_2MW-eff-pur} and~\ref{fig:dune-nd_2MW-energy-error} assume optimised beam intensity of \SI{2}{\mega\watt}.
Efficiency and purity distributions are plotted in Figure~\ref{fig:dune-nd_2MW-eff-pur} as the fraction of \Pgpz photons reconstructed with the respective efficiency/purity.
It can be seen, that purity significantly improves under the assumptions described above.
In particular, taking the mentioned lower and upper limits for pile-up, \SIrange{50}{60}{\percent} of the photons experience no pile-up at all.
While these distributions give a general idea of pile-up, there usefulness for an assessment from a physics point of view is limited.
Therefore, the actual impact on the reconstructed neutrino energy as a function of the true neutrino energy was calculated and plotted in Figure~\ref{fig:dune-nd_2MW-energy-error}.
The y-axis represents the fractional error on the reconstructed neutrino energy and the colors correspond to the same selections as for the purity/efficiency plot in Figure~\ref{fig:dune-nd_2MW-eff-pur}.
However, only misidentified energy from other events affects the total reconstructed energy of the neutrino which is why the magenta selection, including misidentified energy from the same event, is not applicable to this calculation.
It can be seen, that the effect of missed energy starts at a few~\si{\percent} for the lower neutrino energies and the quickly goes down to stabilise around \SI{1}{\percent}.
This indicates that the cone-cylinder union algorithm, even though quite primitive, performs reasonably well.
The error due to misidentified energy starts at around \SI{10}{\percent} and then, settles between \SIrange{1}{2}{\percent}.

\begin{figure}[htb]
	\centering
	\includegraphics[width=\textwidth]{pile-up/10MW/energy_error}
	\caption{Fractional error on the reconstructed neutrino energy due to missed or misidentified energy depositions as a function of true neutrino energy.
	Misidentified energy is shown for three different selections: total deposited by other events (cyan); deposited by other events excluding muons (red), deposited by photons and neutrons from other events only (blue).
	The simulated beam intensity is \SI{10}{\mega\watt} at \SI{80}{\giga\electronvolt} proton energy.}
	\label{fig:dune-nd_10MW-energy-error}
\end{figure}

To get a rough idea of the performance of a 2D wire readout in the same environment, the same study was performed ingoring the Y coordinate completely, leaving everything else untouched.
Of course, this is a gross underestimation of the capabilities of existing reconstruction algorithms for 2D charge readout data.
In particular, contemporary experiments use at least three 2D projections whereas only one was used here.
Even though, doing this comparison serves to show that the simple cone-cylinder union reconstruction algorithm breaks down for two dimensions as can be seen in Figures~\ref{fig:dune-nd_2MW-XZ-eff-pur} and~\ref{fig:dune-nd_2MW-XZ-energy-error}.
The fraction of events not suffering from pile-up is below \SI{10}{\percent} while the error on energy reconstruction has increased to \SIrange{5}{10}{\percent} at high energies and event \SIrange{20}{40}{\percent} at low energies.
Efficiency and energy error due to missed energy are comparable eventhough, the cone and cylinder dimensions are no longer correct for photons not parallel to the XZ plane.

Finally, as a cross-check, the (3D) pile-up study was performed for a hypotheticel \SI{10}{\mega\watt} beam.
As expected, pile-up increases drastically while efficiency does not change much as can be seen in Figures~\ref{fig:dune-nd_10MW-eff-pur} and~\ref{fig:dune-nd_10MW-energy-error}.

In summary, this study shows that even a very primitive EM shower reconstruction algorithm, employing a cone-cylinder union selection, performs well in the high-multiplicity environment of the \dune{} ND, when fed with unambiguous 3D spatial coordinates of energy depositions.
It clearly fails when reduced to two dimensions or presented with a much higher beam intensity.