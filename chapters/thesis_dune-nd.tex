\chapter{Feasibility Study of a Pixelated LArTPC for the \dune Near Detector}
\label{chap:dune-nd}

With \AC\ selected as  the \lar\ component of the \dune\ near detector (ND), there were two main question that needed to be addressed:
\begin{enumerate}
	\item Is a pixelated \lartpc\ feasible?
	\item Can the \lar\ detector handle the high rates?
\end{enumerate}
Number one is addressed in Chapter~\ref{chap:viper}.
This chapter will address question number two.


\section{The High-Multiplicity Environment of the \dune\ Near Detector}
\label{sec:dune-nd_multiplicity}

One of the main goals of \dune\ is to measure the Dirac CP violation phase $\delta_{\m{CP}}$ provided it has a non-degenerate value.
To reach a $3 \sigma$ sensitivity for a \SI{75}{\percent} coverage of the $\delta_{\m{CP}}$ parameter space, an exposure of \SI{1320}{\kilo\tonne\mega\watt years} is required (see Figure~\ref{fig:dune-nd_cpv-sens}).~\cite{dune2}
Assuming the reference design of a \SI{40}{\kilo\tonne} far detector and a \SI{1}{\mega\watt} beam results in a data taking time of \SI{33}{years}.
Therefore, to make the data taking time shorter, a beam $> \SI{1}{\mega\watt}$ is required.
On the other hand, combining the numbers of the reference design in~\cite{dune2}, the event rate in the ND amounts to \SI{0.1}{evt\per\tonne\per\mega\watt} leading to significant event pile-up.
This number does not include rock events---secondary particles produced by beam neutrino interactions in the surrounding material, entering the detector.
The neutrino beam flux is depicted in Figure~\ref{fig:dune-nd_flux}.

\begin{figure}[htb]
	\centering
	\includegraphics[width=.5\textwidth]{dune-nd/dune_cpv75_exp_syst}
	\caption{Expected sensitivity of \dune\ to discovery of CP violation, i.e.\ $\delta_{\m{CP}} \neq\ 0\ \m{or} \pi$ as a function of exposure in \si{\kilo\tonne\mega\watt years}, assuming equal running in neutrino and antineutrino mode.~\cite{dune2}}
	\label{fig:dune-nd_cpv-sens}
\end{figure}

\begin{figure}[htb]
	\centering
	\includegraphics[width=.49\textwidth]{pile-up/FHC_230kA}
	\includegraphics[width=.49\textwidth]{pile-up/RHC_230kA}
	\caption{Neutrino fluxes for the reference beam design operating in neutrino mode (left) and antineutrino mode (right), generated with a \SI{120}{\giga\electronvolt} primary proton beam.~\cite{dune2}}
	\label{fig:dune-nd_flux}
\end{figure}


\section{The Argon Box Simulation Tool}
\label{sec:dune-nd_argon-box}

Argon Box\footnote{\url{https://github.com/dadwyer/argon_box}} was developed with the goal of providing an easy-to-use simulation of particle interactions in \lar\ component of the ND.
At the time of this writing its feature set is quite limited, in particular it does not incorporate any other detector materials except for the box of argon.
Primary particles can either be provided by a particle gun (e.g.\ $e^-$, $n$, $p$, $\mu^+$) or in form of a \emph{HEPEVT} file---a file format standard for passage of particle events between different simulation tools.
For this study, several million neutrino events were produced with the GENIE\footnote{\url{https://genie.hepforge.org}} neutrino event generator.
Secondary particle transport and interaction in Argon Box is performed by Geant4.\footnote{\url{http://geant4.cern.ch}}
Finally, the energy deposition in \lar\ is voxelised and stored together with all the necessary ancillary information about the depositing particle.
The data is stored in the tree format of the ROOT data analysis framework.\footnote{\url{https://root.cern}}
This allows for convenient further processing using ROOT.


\section{$\pi^0$ pile-up study}
\label{sec:dune-nd_pile-up}

A significant amount of $\pi^0$ are produced in several resonant and coherent neutrino interactions (see Section~\ref{sec:nu-detection_interactions} and~\cite{dune2}).
They decay according to
\begin{IEEEeqnarray}{rCl}
	\pi^0 & \qraq & \gamma + \gamma
\end{IEEEeqnarray}
with a branching ratio of \SI{98.8}{\percent}\cite{pdg}.
The photons subsequently produce electromagnetic showers in \lar\ (see Section~\ref{sec:nu-detection_fs}).
At the energies of the \dune\ beam (see Figure~\ref{fig:dune-nd_flux}), most showers do not deposit a homogenous cone of charge but rather a lot of individually resolvable $e^{\pm}$ tracks.
More importantly, there often are significatn gaps in betweend these charge clusters.
A main challenge of shower reconstruction is to associate these well separated charge blobs to the correct event.
Misidentiions lead to a misreconstruction of the neutrino energy.
The resulting discrepancy to the true neutrino energy has the potential to skew the measured energy spectrum and, thus, influence the oscillation measurements.
The complexity of reconstruction paired with the potential impact on the physics measurements makes photons produced by $\pi^0$ decays a good sample to study the robustness to pile-up of a pixelated \lartpc\ in the \dune\ ND environment.

To investigate the effects of pile-up on energy reconstruction, a working reconstruction algorithm is necessary.
However, at the time of writing, official reconstruction tools were only available for \lartpc s read out by wire planes.\footnote{\url{http://larsoft.org}}
Therefore, a rather primitive algorithm for true 3D space points was implemented, under the assumption that a positive outcome of such a pile-up study would imply an even better performance of a more sophisticated reconstruction.
This algorithm is explained in the following.

The basic underlying assumption is that a non-multiplexed pixel readout will yield unambiguous 3D space points of charge deposition with a given resolution, depending on the geometry of the pixel plane, time resolution of the readout electronics, and charge transport effects.
Furthermore, it is assumed that EM shower can be identified and their starting point and direction reconstructed with negligible errors and inefficiencies, i.e.\ this information is taken from the simulation truth.
A cone is calculated in the direction of the shower with its tip at the first charge deposition of the initial photon.
The opening angle and longituginal extent of the cone were optimised by looking at the distributions of the distance from the starting point and angle w.r.t. the direction of the shower.
The finite resolution of the detector is emulated by voxelising the charge deposition with the corresponding resolution in all three spatial coordinates.
This leads to problems near the tip of the cone where the transversal extent is lower than the voxel dimensions.
In particular, it can happen that most of the initial charge is shifted outside the cone.
Furthermore, multiple scattering at lower energies makes the cone model suboptimal near the tip.
Therefore, the acceptance volume for the reconstruction is taken as the union of the cone with a cylinder around the direction of the shower of the same longitudinal extent as the cone.

\begin{table}[htb]
	\centering
	\caption{Parameters of the $\pi^0$ pile-up simulation.}
	\label{tab:dune-nd_pile-up-params}
	\begin{tabu} to \textwidth {|l|S|}
		\hline
		{Beam direction} &			{$Z$ (positive)} \\
		\hline
		{Gravitation} &				{$Y$ (negative)} \\
		\hline
		{Resolution $X$} &			\SI{3}{\milli\metre} \\
		\hline
		{Resolution $Y$} &			\SI{3}{\milli\metre} \\
		\hline
		{Resolution $Z$} &			\SI{3}{\milli\metre} \\
		\hline
		{Target volume $X$} &		\SIrange{-1000}{5000}{\milli\metre} \\
		\hline
		{Active volume $X$} &		\SIrange{0}{4000}{\milli\metre} \\
		\hline
		{Fiducial volume $X$} &		\SIrange{300}{3700}{\milli\metre} \\
		\hline
		{Target volume $Y$} &		\SIrange{-1000}{3500}{\milli\metre} \\
		\hline
		{Active volume $Y$} &		\SIrange{0}{2500}{\milli\metre} \\
		\hline
		{Fiducial volume $Y$} &		\SIrange{300}{2200}{\milli\metre} \\
		\hline
		{Target volume $Z$} &		\SIrange{-4000}{5000}{\milli\metre} \\
		\hline
		{Active volume $Z$} &		\SIrange{0}{5000}{\milli\metre} \\
		\hline
		{Fiducial volume $Z$} &		\SIrange{300}{4700}{\milli\metre} \\
		\hline
		{Detection threshold} &		\SI{0.1}{\mega\electronvolt} \\
		\hline
		{Cone extent} &				{$10 X_{\m{0}} = \SI{1400}{\milli\metre}$} \\
		\hline
		{Cone half opening angle} &	\SI{15}{\degree} \\
		\hline
		{Cylinder radius} &			\SI{25}{\milli\metre} \\
		\hline
		{Beam intensity} &			{$\SI{2}{\mega\watt} \widehat{=} \SI{0.2}{evt\per\tonne}$} \\
		\hline
	\end{tabu}
\end{table}