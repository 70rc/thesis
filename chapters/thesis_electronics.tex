\chapter{Electronics Studies}
\label{chap:electronics}

For a heavy MIP with $\dv{E}{x} \approx \SI{2.1}{\mega\electronvolt\per\centi\metre}$, a \lartpc{} has a charge yield of $\order{\SI{1}{\femto\coulomb\per\milli\metre}}$ as explained in Chapter~\ref{chap:lartpc}.
The readout electronics need to be able to reliably digitise this charge.
This chapter aims to outline the challenges based on present designs and then present several tests of future approaches addressing them.

\section{Existing Chain}
\label{sec:electronics_existing}
Contemporary electronics schemes shall be introduced by looking at the existing readout chain at LHEP at the University of Bern.
It was originally designed for the \AT{} experiment and a more detailed description can be found in~\cite{AT_larasic}.

The charge collected by the readout plane is amplified by LARASIC4*~\cite{larasic} cryogenic charge amplifiers developed by Brookhaven National Laboratory (BNL) for \uboone{}~\cite{uboone}.
A performance characterisation of these application-specific integrated circuits (ASICs) can be found in~\cite{AT_larasic}.
Their main features include

\begin{itemize}
	\item \num{16} channels per ASIC;
	\item low noise charge amplifiers incorporating high-order filters;
	\item per channel programmable gain of \SIlist[list-final-separator = { or }]{4.7; 7.8; 14; 25}{\milli\volt\per\femto\coulomb};
	\item per channel programmable filter peaking time of \SIlist[list-final-separator = { or }]{0.5; 1.0; 2.0; 3.0}{\micro\second};
	\item built-in test capacitance connected to dedicated external test pulse input for calibration;
	\item and a power dissipation \SI{< 10}{\milli\watt} per channel.
\end{itemize}

\begin{figure}[htb]
	\centering
	\includegraphics[width=\textwidth]{viper/ReadoutChain_old}
	\caption{Readout chain used for the pixel test. The picture of the LARASIC cryogenic front-end preamplifiers shows them installed in an older wire readout setup.~\cite{AT_larasic}}
	\todo[inline]{better ASIC picture}
	\label{fig:viper_readoutChain_old}
\end{figure}

The cryogenic preamplifiers are mounted as close as possible to the readout in order to minimise noise pick-up on these very sensitive lines.
Via an inter-integrated circuit (I$^2$C) bus, LARASICs can be programmed to the different aforementioned configurations.
For this purpose, they are connected to a bespoke NIM module housing an Arduino which generates the I$^2$C signals, a test pulse generator, and multiple low-noise voltage regulators providing power to the LARASICs.
The output of the preamplifiers is fed to buffer amplifiers mounted on top of the signal feedthrough by means of flexible Kapton ribbon cables.
The buffers operate at room temperature, have a unity gain, and match the output impedance of the LARASICs to the \SI{50}{\ohm} input impedance of the downstream digitisers.
From the buffers, the signals are routed via \SI{50}{\ohm} unbalanced coaxial lines to \emph{CAEN V1724}\footnote{\url{http://www.caen.it}} \SI{14}{bit} digitisers mounted in a VME crate.
For debugging purposes, the output of the buffers can be routed to an oscilloscope via a coaxial T-piece.
Finally, the digital data is read out from the VME crate via a fibre-optic link by a standard PC.
Figure~\ref{fig:viper_readoutChain_old} depicts the entire readout chain.
The complete analogue signal path from the pixel plane to the VME digitisers is single-ended and thus prone to ground loops and all associated noise problems.

\begin{figure}[htb]
	\centering
	\missingfigure{Event from the first measurement campaign of the pixel prototype.}
	\caption{Event from the first measurement campaign of the pixel prototype.}
	\label{fig:viper_noisy-event}
\end{figure}

During the first pixelated readout measurement campaign, it became apparent that the data was significantly impaired by noise.
As can be seen in Figure~\ref{fig:viper_noisy-event}, the noise amplitude is similar over multiple channels.
This implies a common mode component that cannot originate from inductive pick-up.
Instead, the noise is likely generated by self-oscillating parts of the signal path due to ground loops and parasitic impedances.
For the second measurement campaign, different steps were take to mitigate this behaviour through modifications to detector location, power supply, signal path, and intrinsic capacitance.

A correlation between noise levels and the running state of the air condition in the utility room next to the lab was found.
Therefore, the experimental setup was moved away from the wall facing the utility room.

A decoupled clean power grid was built in the lab.
A Motor Generator (M-G set) separates the lab grid mechanically from the building power supply.
Thus, any noise present on the latter is prevented from entering the experimental setup.
Furthermore, this decouples the lab grid entirely from the building ground preventing ground loops via electric mains.

The signal path from the impedance-matching buffer amplifiers to the digitisers---i.e. the warm signal path---was changed from single-ended to differential signalling.
For conventional single-ended signalling, the signal is measured as the voltage or current difference between a signal conductor and a ground common to the signal source and the signal sink.
Using a common ground as signal return path can have several undesired effects.
To shield the signal conductor, it is usually enclosed in a ground shield.
If the latter is connected on both sides, a ground loop can result for instance in combination with a shared power supply ground.
Ground loops can pick up noise through induction if the resistance along the loop is high enough.
A second way to couple noise into a single-ended system is by shifting the potential on the common ground away from the reference voltage or current, for instance due to high currents flowing through a lossy ground connection.
Because the signal is always measured against the common ground, it will be distorted.
In differential signalling, the signal is not measured between a signal conductor and ground but instead between two signal conductors.
This works by putting an inverted waveform of the signal on a second conductor.
The signal is recovered by taking the difference between to two signal conductors.
As a result, the signal sink needs not be connected to the same ground as the signal source because the signal is independent of ground.
Ground loops can thus be avoided in the signal path.
Furthermore, the effects of noise pick-up on the signal lines is drastically reduced.
Due to the completely symmetric signal path, inductive noie pick-up is equal on both signal conductors as opposed to single-ended signals where the signal path is not symmetric.
In the signal sink, the difference between the two symmetric signal conductors is formed and everything that is present on both of them, such as the inductively picked up noise, cancels out.
In the pixel prototype setup, differential signalling was realised by replacing the buffer amplifiers by single-ended to differential amplifiers and inserting another stage upstream of the digitisers to change the signal back to \SI{50}{\ohm} single-ended, matching the input of the digitisers.
Like this, noise pick-up outside the cryostat could be reduced as well as sensitivity to ground loops between the detector and the DAQ rack.

A source of noise was identified in the layout of the pixel readout plane.
It was found that due to several ground planes and long tracks in the PCB, parasitic capacitances were very high, in particular for pixel channels.
This is problematic because for high enough frequencies---determined by $RC$---, the input is shorted to ground creating a ground loop again.
Through this capacitive coupling to ground, the system can start to oscillate.
One evidence for this is that the noise is equal over multiple channels, so-called common-mode noise.
More specifically, the noise is equal over groups of channels.
Investigating this, it was found that these groups correspond to channels of roughly equal parasitic capacitance.
Also, the noise amplitude is higher on channels with higher capacitance.
To solve this problem, the PCB design was optimised by removing unnecessary ground planes, routing signal tracks outside necessary ground planes and increasing the thickness of the PCB.

\begin{figure}[htb]
	\centering
	\missingfigure{Event from the second measurement campaign of the pixel prototype after improving the readout chain.}
	\caption{Event from the second measurement campaign of the pixel prototype after improving the readout chain.}
	\label{fig:viper_good-event}
\end{figure}

As can be seen from Figures~\ref{fig:viper_noisy-event} and~\ref{fig:viper_good-event}, there is a significant decrease in noise after commissioning all of the above improvements to the readout chain.
This can also be seen from Figures~\ref{fig:viper_snr-noisy} and~\ref{fig:viper_snr-good} depicting the signal to noise ratio of the two measurement campaigns.

\begin{figure}[htb]
	\centering
	\missingfigure{Signal vs noise using the old readout chain.}
	\caption{Signal vs noise using the old readout chain.}
	\label{fig:viper_snr-noisy}
\end{figure}

\begin{figure}[htb]
	\centering
	\missingfigure{Signal vs noise after imrpoving the readout chain.}
	\caption{Signal vs noise after imrpoving the readout chain.}
	\label{fig:viper_snr-good}
\end{figure}


\section{Improved Cold Electronics for Pixelated Readouts}
\label{sec:electronics_pixels}

This section describes the challenges met by electronics for pixelated \lartpc{}s and possible solutions.
As an upgrade to the LARASIC preamplifiers developed for \uboone{}, BNL is developing cold charge readout electronics for the \dune{} far detector.
In particular, the plan is to accompany the cryogenic charge preamplifiers by cryogenic Analogue-to-Digital Converters (ADCs).
As mentioned in Section~\ref{sec:lartpc_electronics}, this can improved noise because of both shorter analogue signal lines and reduced thermal noise of the electronics.
Furthermore, cold digitisation allows multiplexing of the data on high-speed digital links, reducing the number of needed signal cable feedthroughs.

However, designing reliable electronics at cryogenic temperatures is not an easy task.
ADCs in particular are very sensitive to stable reference voltages required for proper analogue to digital conversion.
Another problem arises from the fact that digital electronics in general required clocks.
In general, proper timing requires sharp edges and is usually realised as a square wave.
According to Fourrier analysis, a square wave produces a high level of harmonics.
This is particularly problematic in case of readout wires that act as antennas and can pick up theses clock signals.
Another important aspect is power dissipation.
All power dissipated by cryogenic electronics needs to be compensated for in order to prevent the \lar{} from boiling.
This is particularly problematic for a pixelated readout that requires a much higher number of readout channels than a wire readout (see Chapter~\ref{chap:charge-ro}).

In the course of this work, the cryogenic ADC ASICs developed by BNL were evaluated to be used in the near detector as well.
The author joined the team at BNL in cold tests of the devices.
One of the results of these tests is presented here to illustrate the difficulties of cryogenic ADCs.
As a disclaimer, it should be noted that this is by no means the current status of the ADCs at the time of this writing.
The described tests were performed in the Fall of 2016 at BNL.

The tests were focused on the linearity, an important characteristic of an ADC.
It describes the relation between the applied inpurt voltage and the calculated digital number at the output.
In the case of the BNL ADCs, this relation is expected to be strictly linear.
To test this, a voltage ramp is applied to the input and the converted digital values are recorded.

\begin{figure}[htb]
	\centering
	\includegraphics[width=\textwidth]{bnl/bnl_adc_lin}
	\caption{Linearity measurement of the BNL cryogenic ADC ASICs with input voltage on the x-axis and ADC value or code on the y-axis.
	Color represents the number of measurements.
	The measurements were performed in liquid nitrogen.}
	\label{fig:bnl_adc_lin}
\end{figure}

A typical measurement is shown in Figure~\ref{fig:bnl_adc_lin}.
The expected shape is one straight diagonal line from the bottom left to the top right corner, i.e.\ a linear relationship between input voltage and ADC value or \emph{code}.
Two particular deviations from this are visible: gaps accompanied by horizontal lines and a wobbly response around zero.
Upon close inspection, it can be seen that the gaps have the same voltage range as the horizontal lines.
The meaning of this is that for this input voltage range, the ADC output is \emph{stuck} at the same value.
Both these effects mean that the detector response to detected charge and thus energy deposition is not linear.
While some non-linearities can be compensated in offline data analysis, this is not possible for the sticking ADC values because they correspond to a range of input voltages.
This impairs the energy resolution of the detector.

The explanation for the non-linearities are related to the electronic design of the ASIC.
Additionally, the issue was not fully understood at the time of these tests.
Therefore, it is out of the scope of this work and not given here.
The measurements are shown to illustrate the difficulties of designing a reliable cryogenic ADC.

Leaving aside the non-linear response, the BNL ADC ASICs are not suitable for use in conjunction with a pixelated \lartpc{} charge readout.
Being designed for wire readouts, no strong focus was laid on power dissipation which is $\approx \SI{5}{\milli\watt}$ per channel.
Combined with the one of the LARASIC (\SI{10}{\milli\watt}), a total of \SI{15}{\milli\watt} is dissipated.
For a pixelated \dune{} ND with $\order{\num{e7}}$ channels, the resulting required cooling power would be \SI{150}{\kilo\watt}.\todo{sauce}

\todo[inline, color=red]{BNL}
\todo[inline, color=red]{LArPix}