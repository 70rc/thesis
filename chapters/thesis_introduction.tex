\renewcommand{\Chapter}{Introduction}
\chapter*{\Chapter\label{chap:introduction}}
\chaptermark{\Chapter}
\addcontentsline{toc}{chapter}{\Chapter}
To measure the remaining unknown parameters of the PMNS matrix and the mass hierarchy in neutrino oscillation physics, detectors with masses of orders of magnitude larger than so far are needed.
There are two obvious demonstrated choices.
Water Cherenkov detectors provide a large cost-efficient target mass which makes them interesting for proton decay searches as well.
A \lartpc\ on the other hand is an imaging detector providing precise tracking, while simultaneously acting as a homogeneous calorimeter with a precise measurement of energy deposited.
Furthermore, the 3D resolution of these detectors allows for an efficient $e$ $\gamma$ separation, thus reducing the $\pi^0$ background in $\nu_e$ identification.

For a TPC to work, the detection medium needs to have a low electronegativity.
Noble gases are an obvious choice for this.
To provide a high target mass, they are liquified.
The ideal choice would be LXe which therefore is used for low volume experiments such as neutrino-less double beta decay and dark matter searches.
The drawback of xenon is that it is expensive and rare.
Besides, \lar\ has a longer radiation length than LXe and produces a comparable number of photons per \si{\mega\electronvolt}, it is also abundant and relatively cheap.

Future neutrino oscillation physics experiments aim to measure the $\delta_{\m{CP}}$ phase and determine the neutrino mass hierarchy.
With the recent measurement of the $\theta_{13}$ mixing angle, new experiments can be tuned to the right parameter space to be most sensitive in the most probable regions.
Both measurements require much higher statistics than contemporary experiments have achieved.
In turn, higher statistics require that the neutrino flux is increased, and an increase in detection efficiencies.
Efficiency is improved by increasing the size of the detector by up to two orders of magnitude in comparison to the current generation, the increase in size also improves event containment.

These imposed requirements lead to several new challenges for the detector design.
Higher flux beams put higher demands on the precision of the trigger system to reach reasonable trigger efficiencies and purities.
Additionally, \lartpc s have some intrinsic issues in high-flux particle beams, most of these issues stem from the detector volume required for the containment of events:
The long drift-times lead to event pile-up and also require higher voltages and higher purity argon, and unconfined scintillation light makes it harder to use a light detection system as a coincidence trigger.
Event pile-up is worsened still by the intrinsic reconstruction ambiguities of the conventional wire readout.
The complexity of event reconstruction due to the projective nature of wires further reduces the ability of the trigger system.
As well as being notoriously fragile, wire readouts and their associated framework are a major limiting factor in detector design.
The most inhibitive drawback to monolithic detectors is the down-time cost involved in repair or upgrade work.
Independent of future neutrino experiments, the solution of all those problems will lead to next generation l\lartpc s bringing huge improvements for all future applications of such detectors.