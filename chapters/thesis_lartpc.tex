\chapter{The Liquid Argon Time Projection Chamber (\lartpc)}
\label{chap:lartpc}

The Time Projection Chamber (TPC) is a derivative of Charpak's Multi-Wire Proportional Chamber (MWPC)\cite{mwpc} developed by Nygren in the late 1970s~\cite{nygrenTPC}.
Crossing charged particles ionise the detection medium which was gaseous in the original design.
To prevent the recombination of the ions and electrons, an electric field is applied.
In this field, the electrons drift towards a two-dimensional readout plane (an MWPC in the original design).
The charge readout is triggered by a scintillation light readout which also provides accurate timing of an event.
Using this, one can measure the time for the ionisation electrons to reach the readout plane.
As the drift speed of charged particles in the detection medium is constant and if it is known, the coordinate in drift direction can be calculated from the drift time.

While gaseous TPCs already provide very accurate tracking, they have the disadvantage that the target mass and thus, the cross-section of the detection medium is quite low resulting in a low interaction rate.
Therefore, in 1977, Rubbia proposed to use liquid argon as a detection medium~\cite{lartpc}.
This requires a cryogenic detector while gaseous detectors can be operated at room temperature.


\section{Liquid Argon (\lar) as a Detection Medium}
\label{sec:lartpc_lar}

For an efficient particle detection by a TPC, several properties of the sensitive medium are of interest, such as ionisation and light yield, electron-ion pair recombination, dielectric strength, length scales of electromagnetic and hadronic interactions, density, transparency to its own scintillation light, and the boiling point.
\lar\ is quite unique as it has all the necessary properties while at the same time it is comparably cheap because it is readily available in the Earth's atmosphere.
A summary of its properties can be found in Table~\ref{tab:lartpc_larprop}.
Xenon for instance slightly surpasses argon in many aspects but is prohibitively expensive to build large detectors.
A boiling point of $\approx \SI{87}{\kelvin}$ raises the need for strong thermal insulation and a potent cooling system for \lar , though the requirements are far less stringent then for liquid helium.
This section will outline the most important \lar\ properties.

\begin{table}[htb]
	\centering
	\caption{Properties of \lar\ taken from~\cite{NobleGasDetectors} where not specified otherwise}
	\label{tab:lartpc_larprop}
	\begin{tabu} to \textwidth {|ll|Ss|}
		\hline
		{Molar mass} &									$\mu$ &					3.9948e1 &	\gram\per\mol \\
		\hline
		{Boiling point at} \SI{1.01325e5}{\pascal} &	$T_{\m{S}}$ &			8.726e1 &	\kelvin \\
		\hline
		{Density at} $T_{\m{S}}$ &						$\rho_{\m{S}}$ &		1.399e3 &	\kilo\gram\per\cubic\metre \\
		\hline
		{Dielectric constant~\cite{dielConst}} &		$\varepsilon_{\m{r}}$ &	1.504 &		\\
		\hline
		{Required energy per electron-ion pair} &		$W_{\m{i}}$ &			2.36e1 &	\electronvolt \\
		\hline
		{Required energy per photon} &					$W_{\m{sc}}$ &			1.95e1 &	\electronvolt \\
		\hline
		{Fano factor} &									$F$ &					1.07e-1 &	\\
		\hline
		{Electromagnetic radiation length} &			$X_{\m{0}}$ &			1.4e-1 &	\metre \\
		\hline
		{Hadronic interaction length} &					$\lambda_{\m{int}}$ &	8.37e-1 &	\metre \\
		\hline
		{Peak scintillation wavelength} &				$\lambda_{\m{scint}}$ &	1.28e-7 &	\metre \\
		\hline
		{Scintillation attenuation length} &			$\lambda_{\m{att}}$ &	6.6e-1 &	\metre \\
		\hline
		{Concentration in air by volume} &				&						9.34e-1 &	\percent \\
		\hline
	\end{tabu}
\end{table}

In order to register the ionisations tracks of charged particles in a TPC, two processes are crucial: production and transportation of charge.
The charge production needs to be high enough to be detectable by the available electronics.
This is given by the energy required to produce an electron-ion pair $W_{\m{i}}$.
The $W_{\m{i}}$ value of \SI{23.6}{\electronvolt} for \lar\ is challenging but manageable with contemporary electronics as shown in Section~\ref{sec:lartpc_electronics}.
Naturally, this imposes a lower limit on detectable $\frac{\m{d}E}{\m{d}x}$.

Charge transport is influenced by multiple factors.
The ultimate goal is to collect as much of the produced charge as possible.
Recombination is the main process opposing this.
While it can be partially mitigated by increasing the electric field, it cannot be eliminated completely.
Even if that was possible, it would not be beneficial because the scintillation light needed for the drift time measurement is partly produced by recombining electron-ion pairs.
The relation between the strength of the drift field and charge yield can be described by the \emph{box model}~\cite{box-model}.
It assumes that the ion-electron pairs are isolated and initially uniformly populate a box of a given size.
Furthermore, the diffusion of electrons and ions as well as the ion drift velocity (\num{1e5} times smaller than for electrons) are assumed to be negligible.
For a produced charge $Q_{\m{0}}$ and a collected charge $Q$, the collection ratio is given by
\begin{IEEEeqnarray}{rCl}
	\frac{Q}{Q_{\m{0}}} & = & \frac{1}{\xi} \ln(1 + \xi) \m{,}
\end{IEEEeqnarray}
with a parameter $\xi$ depending on the drift field, electron mobility, initial number of ion-electron pairs, chosen size of the box and recombination coefficient.
Figure~\ref{fig:lartpc_box-model} shows a measurement by LHEP of the collected charge in an \SI{8}{\milli\metre} drift \lartpc\ for different drift field intensities and nitrogen concentrations.

\begin{figure}
	\centering
	\includegraphics[viewport=14 5 511 351, clip, width=\textwidth]{lartpc/box-model}
	\caption{Collected charge in an \SI{8}{\milli\metre} drift \lartpc\ as a function of electric field, at various levels of nitrogen concentration.
	The lines represent box model fits.~\cite{grna-lhep}}
	\label{fig:lartpc_box-model}
\end{figure}

A second process affecting charge transport is electron trapping by impurities; the probability of an electron becoming attached to an atom in the medium.
For the argon itself, this is highly unlikely because its outer electron shell is fully populated.
This is one or the reasons why (liquefied) noble gases are a prime choice for TPCs.
Nevertheless, drifting electrons can be captured by impurities in the argon.
Oxygen is particularly bad due to its high electronegativity.
It is, therefore, important to maintain purity of the detection medium.

The velocity of the charge drifting in an electric field is related to the so-called mobility, $\mu$, by

\begin{IEEEeqnarray}{rCl}
	\va{v} & = & \mu\qty(\va{E}) \va{E} \m{,}
\end{IEEEeqnarray}

where $\mu$ in general depends on the electric field.
This means, the higher the field is, the higher the charge velocity and thus, the lower the drift time.
One wants to keep drift times low for multiple reasons.
One of them are the aforementioned impurities.
They cause the charge to posses a finite lifetime following an exponential decay with time.
A consequence is that higher impurities can be partially compensated by a higher field.
In a beam experiment with a given beam timing, increasing the drift time will increase pile-up, i.e. the number of events simultaneously present in the detector.
Pile-up in turn, complicates event reconstruction.
On the other hand, the readout electronics need to be fast enough to guarantee the required spatial resolution in the drift coordinate which defines an upper limit for the drift velocity.
A reasonable value from a purity point of view is a drift time of $\order{\SI{1}{\milli\second}}$.
For a detector size of $\order{\SI{1}{\metre}}$, the required drift speed is $\order{\SI{1}{\milli\metre\per\micro\second}}$ which requires a field of $\order{\SI{1}{\kilo\volt\per\centi\metre}}$.

For detectors much larger than \SI{1}{\metre}, a drift field of \SI{1}{\kilo\volt\per\centi\metre} becomes challenging due to the required high cathode voltage.
Soon after entering \lartpc\ R\&D, the Bern group realised that the reported dielectric strength of \lar\ is much lower~\cite{breakdown_14} than measured by Swan et al.\ in 1960~\cite{swan1, swan2}.
It turned out, opposing the assumption of Swan et al., that the dielectric strength is not independent of the absolute dimensions of the electrodes.
This led to a very detailed study of breakdowns in \lar\ in the course of this thesis which is presented in Chapter~\ref{chap:hv}.


\begin{itemize}
	\item General specs
	\item Electronegativity
	\item Ionisation
	\item Scintillation
	\item Recombination
	\item Diffusion
	\item Dielectric strength
\end{itemize}
\cite{NobleGasDetectors}


\section{Electric Field Generation}
\label{sec:lartpc_efield}

For charge separation and drift, an electric field of $\order{\SI{1}{\kilo\volt\per\centi\metre}}$ is needed inside the fiducial volume of a \lartpc.
An easy way to achieve this is by means of field shaping rings fed by a resistive divider between cathode and anode.
The drawback is the need for a feedthrough capable of withstanding the full cathode voltage.
One alternative is to generate the high voltage inside the cryostat, for instance using a Greinacher voltage multiplier circuit as the one used for the \AT\ experiment at LHEP at the University of Bern~\cite{AT}.
A Greinacher multiplier works by pumping up a cascade of capacitors and diodes using a high frequency source.
However, while the voltage generation itself worked well, this approach proved to be impractical because the high frequency voltage needed to charge the multiplier interfered with the readout and therefore had to be turned off during data-taking.
Recharging, in turn, caused a lot of detector down time.


\section{Charge Readout}
\label{sec:lartpc_charge-ro}

Classically, the charge readout of a liquid argon TPC is done using wires with a diameter of $\order{\SI{.1}{\milli\metre}}$.
One wire plane delivers a 2D projection of the ionisation tracks in the detection medium.
This has two consequences:
\begin{enumerate}
	\item At least two parallel wire planes are needed to be able to reconstruct the 3D event topology.
	\item In theory, the higher the complexity of the event, the more planes would be required to be able to fully reconstruct it.
\end{enumerate}
Multiple wire planes can be realised by operating only the last one (in drift direction) in charge collection mode.
All the preceeding wire planes are biased in a such a way that they are transparent to the incoming charge but pick up an induction signal during the passage of the latter.
A typical number of wire planes for currently operational detectors is three, tilted at \SI{60}{\degree} to each other.


\section{Light Readout}
\label{sec:lartpc_light-ro}

In order to calculate the distance the charge has drifted along the electric field (i.e.\ the space coordinate perpendicular to the readout plane), one needs to record the drift time.
The data acquisition can record the time of the arrival of the charge at the readout plane.
What is missing is the time of the charge production.
This can be acquired by registering the scintillation light produced alongside the ionisation of the detection medium.
Contemporary detector designs employ PhotoMultiplier Tubes (PMTs) for this purpose.
PMTs are a well established technology with a high quantum efficiency and a fast response.

The light impinging upon a PMT is converted to an electron by a photo cathode coated on the sensitive surface of the PMT.
These photo cathodes have a limited absorption spectrum.
In particular, the scintillation light of \lar\ does not fall inside this spectrum for most photo cathodes.
That is why the vacuum UV scintillation light needs to be converted to the visible range where it can be efficiently detected by a PMT.
A widespread wavelength-shifter capable of achieving this is TetraPhenyl Butadiene (TPB).
Therefore, a common setup consists of either coating the PMT~\cite{icarus}, or a surface in front of it~\cite{uboone} with TPB.


\section{Electronics}
\label{sec:lartpc_electronics}

Using $W_{\m{i}}$, the energy required to produce one electron ion pair, from Section~\ref{sec:lartpc_lar} and assuming a MIP, one gets

\begin{IEEEeqnarray*}{rCl}
	\dv{Q}{x}	= \frac{\eval{\dv{E}{x}}_{\m{MIP}}}{W_{\m{i}}} \si{\elementarycharge}
			&	= & \frac{\SI{210}{\kilo\electronvolt\per\milli\metre}}{\SI{23.6}{\electronvolt}} \si{\elementarycharge}
				\approx \SI{8900}{\elementarycharge\per\milli\metre}
				\approx \SI{1.4}{\femto\coulomb\per\milli\metre}
\end{IEEEeqnarray*}

as a rough estimation for the charge yield.
This calculation does not incorporate recombination, diffusion and charge lifetime, meaning that in a real experiment the value will be even lower.
The result is that the readout electronics need to be capable of detecting $\order{\si{\femto\coulomb}}$ charges.

That is why the charge signal needs to be amplified before digitisation.
This is achieved by means of an integrating amplifier which also converts the charge to a voltage.
Early \lartpc\ designs used preamplifiers outside the cryostat at room temperature for this.
From a noise point of view though, it is beneficial to put the amplifiers inside the cryostat submerged in \lar\ for two reasons.
Firstly, the closer to the source the amplifier is located, the shorter the low-signal lines will be, resulting in less pick-up noise.
Secondly, the temperature-dependent Johnson-Nyquist noise of the amplifiers will be reduced at cryogenic temperatures.

For the same reasons it makes sense to operate the entire analogue signal chain at cryogenic temperatures.
This would also help to eliminate ground loops which can pick-up noise inductively or provoke self-oscillation of the analog signal circuitry.
On the other hand, in general it is not easy to operate a circuit at cryogenic temperatures.
Usually, a complete redesign of the circuit is necessary due to most components operating outside their guaranteed temperature range.
For some complex active components like the amplifiers and ADCs, even a redesign of the integrated circuit might be necessary.
On the other hand, placing the digitisers too close to the readout might result in elevated noise levels due to the digital clocks coupling into the analogue signal path.

The requirements on the electronics are given by the required sensitivity of the detector.
The necessary bit depth of the ADCs is given by the required dynamic range, i.e.\ the minimum and maximum amount of charge the readout needs to be able to register.
While the spatial resolution in the two coordinates parallel to the readout plane is given by the pitch of the electrodes, the accuracy of the third coordinate is given by the timing accuracy.
This in turn depends on three properties: the timing accuracy of the light readout, the sampling time of the ADCs and the peaking time of the preamplifiers.
Peaking time is the time needed until the output of the preamplifier reaches its maximum (peak) for a delta pulse input.


\section{Challenges of Future Detectors}
\label{sec:lartpc_challenges}

To accomplish the physics goals of future neutrino detectors, outlined in Chapter~\ref{chap:introduction}, much higher statistics than with today's experiments are necessary.
There are two obvious ways to do this: Increase the beam flux and detector size.
Scaling up a \lartpc\ brings several challenges, in particular for the drift HV and wire readout planes.

At the same drift field, the cathode voltage scales with the size of the detector in drift direction.
This in turn increases the required distance of the cathode from all grounded parts.
Where the cathode is close to the \lar\ vessel, this inevitably leads to more dead volume that cannot be used for particle detection.
HV problems are worsened by the fact that an increased drift distance also means an increased drift time at the same field.
This can either be compensated by increasing the charge lifetime and thus the \lar\ purity accordingly or by increasing the drift speed and thus the drift field.
In summary, at the same \lar\ purity, the cathode voltage scales more than linearly with detector size in drift direction.

Further problems are associated with the classic wire readouts employed in \lartpc s, such as mechanical construction and event pile-up.
One of the mechanical requirements on a wire readout is that it should be as planar as possible.
Sagging wires caused by insufficient mechanic tension lead to distortions in spatial reconstruction.
For large detectors, possessing thousands of wires on a single frame, this becomes quite challenging.
Every wire that has a slight deviation in tension from its neighbours will start to sag.
This is worsened by the fact that the construction needs to withstand extreme temperature gradients during detector cool-down.

The second problem of wires, event pile-up is a consequence of the increased flux required for future experiments.
It is rooted in the way event reconstruction works for wires.
As mentioned above, effectively, wire planes do not produce real 3D event topologies but rather multiple 2D projections.
In order to achieve true 3D events, they need to be disentangled from the 2D projections.
If an event is complex enough, this cannot be done unambiguously with a limited number of 2D projections.
This problem is especially serious in case of a near detector.
The envisioned \dune\ near detector, for instance, is expected to see a total of \num{3.14e5} neutrino events per tonne of Ar for \SI{1e20}{POT} (protons on target).~\cite{dune2}
Scaling down this number to one beam spill (\SI{1e14}{POT})\footnote{Assuming a \SI{2}{\mega\watt} beam reaching \SI{2e21}{POT\per a} with a spill cycle of \SI{1.2}{\second} and \SI{66.7}{\percent} beam up-time.~\cite{dune2, dune4}} leads to $\approx \num{0.3}$ neutrino events.
For a \SI{30}{\tonne} near detector, this would result in \num{9} neutrino events per beam spill excluding any additional cosmic-induced events.

On top of the event reconstruction problems, event pile-up also poses a challenge for trigger accuracy.
In a monolithic detector, the scintillation light produced alongside the ionisation charge scatters across a large volume triggering a big portion of the light readout system.
Thus, matching a scintillation flash to the corresponding charge to get the correct timing of the event is a non-trivial task.