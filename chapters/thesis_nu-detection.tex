\chapter{Neutrino detection\label{chap:nu-detection}}
Neutrino physics has seen massive progress from first detection \num{60}~years ago to planned billion dollar experiments in the near future.
This chapter shall give an overview of the history of neutrino detectors, describe the current state of the field and then introduce the most relevant physics.


\section{History}
This section puts focus on the evolution of neutrino detection over time.
Giunti and Kim provide a more detailed history of neutrino physics.~\cite{giunti}

Wolfgang Pauli in 1930 proposed a new neutral, weakly interacting fermion to the \emph{Radioactive Ladies and Gentlemen} which he called the \emph{neutron}.
It should explain the continuous energy spectrum measured for $\beta$-decays by Chadwick.
However, the same Chadwick in 1932 discovered the particle we call neutron today.~\cite{neutron}
Upon this, Fermi proposed the name \emph{neutrino} and a little later came up with a new theory for $\beta$-decay.~\cite{betaDecay}

It took almost another quarter of a century until the neutrino was experimentally detected for the first time by Reines and Cowan in 1956.~\cite{reinesCowan}
They built a detector for the reaction
\begin{IEEEeqnarray}{rCl}
	\label{eq:nu-detection_reinesCowan}
	\overline{\nu}_e + p & \rightarrow & e^+ + n
\end{IEEEeqnarray}
and put it next to the nuclear reactor of the Savannah River Plant in South Carolina, US.
It consisted of two water tanks sandwiched in between three liquid scintillator tanks with photomultiplier tubes (PMTs) on the sidewalls.
The water was the target to induce the above reaction while the scintillator tanks had the task to detect the resulting positron and neutron.
A free positron will slow down in matter and eventually get captured by a shell electron which will produce two back-to-back $\gamma$s with an energy of \SI{511}{\kilo\electronvolt} each.
These will produce scintillation light in the two adjacent tanks and thus can be detected by forming a coincidence of the PMTs of the two tanks.
Neutron detection is achieved by doping the water target with cadmium which captures the free neutrons producing multiple $\gamma$s that can again be detected using the coincidence of the two adjacent scintillator tanks.
The neutron capture is much slower than the positron one.
Therefore, the process from Equation~\eqref{eq:nu-detection_reinesCowan} produces a very distinct signal in the detector consisting of a lower pulse from the positron capture and a higher pulse from the neutron capture a few \si{\micro\second} later.
Backgrounds can be efficiently rejected employing this technique.
That drawback is that detection is limited to the $\overline{\nu}_e$ interaction in Equation~\eqref{eq:nu-detection_reinesCowan}.

In 1962, Lederman et al.\ proved the existence of the $\nu_{\mu}$ at the Alternating Gradient Synchrotron (AGS) at Brookhaven National Laboratory (BNL) in New York, US.~\cite{numu}
For the first time, they produced $\nu_{\mu}$s using an accelerator.
The protons from the AGS were guided onto a beryllium target producing pions which in turn decay according to
\begin{IEEEeqnarray}{rCl}
	\label{eq:nu-detection_pion-decay}
	\pi^+ & \rightarrow & \mu^+ + \overline{\nu}_{\mu} \\
	\pi^- & \rightarrow & \mu^- + \nu_{\mu}
\end{IEEEeqnarray}
producing a beam of muon (anti)neutrinos.
Spark chambers were used to detect the neutrinos.
They were placed behind a \SI{13.5}{\metre} wall of iron shielding used to stop the muons and remaining hadrons from the beam.

A spark chamber consists of several parallel conducting plates immersed in a counting gas, typically a mixture of helium and neon.
Every other plate is connected to a pulsed high-voltage power supply while the rest is grounded.
At either end of the stack, triggering detectors (usually scintillators coupled to PMTs) are placed.
When two coinciding signals from these are received, a high-voltage pulse is applied to the plates.
If this happens fast enough ($\order{\SI{10}{\micro\second}}$), a spark forms along the electric field lines where the counting gas has been ionised by the incident particle(s).
The amplitude and the duration of the HV pulse need to be carefully tuned in order to reach the threshold of spark formation but to prevent random sparks on sharp edges and spacers etc.
A gas amplification of \numrange{1e8}{1e9} is required to achieve this.
Furthermore, the rising edge of the HV pulse needs to be extremely short ($\order{\SI{1}{\nano\second}}$).
If it was too long, it would drift the ionised track towards the electrodes before the field is high enough to initiate a discharge.
Switching high voltages at this speed is not easy.
Additionally, spark chambers have quite high dead times of $\order{\SI{100}{\milli\second}}$ which is needed for the ionisation charge to clear.
A \emph{clearing field} or an electronegative quenching gas additive can be used to speed up this process.~\cite{grupen}

After Davis failed to measure the lepton-number-violating reaction
\begin{IEEEeqnarray}{rCl}
	\label{eq:nu-detection_homestake}
	\overline{\nu}_e + \ce{^{37}Cl} & \rightarrow & \ce{^{37}Ar} + e^- \m{,}
\end{IEEEeqnarray}
he decided to replace the $\overline{\nu}_e$ by solar $\nu_e$.~\cite{homestake68, homestake98}
Surprisingly, they measured a flux approximately on third lower than predicted by the standard solar model (SSM).
This became famous as the solar neutrino problem only to be resolved in 2002 by the SNO experiment.
Davis' experiment was located \SI{1478}{\metre} (\SI{4200}{\metre} water equivalent) underground in the Homestake gold mine at Lead, South Dakota, US.
The detector consisted of a tank filled with \SI{615}{\tonne} of tetrachloroethylene, \ce{C_2Cl_4}.
As opposed to the two experiments above, this was a \emph{radiochemical} detector which can detect neutrino interactions only offline.
According to Equation~\eqref{eq:nu-detection_homestake}, an incident neutrino converts one of the chlorine atoms in the detector into an unstable argon isotope.
After exposure, the tank is purged by pumping helium gas through the liquid which extracts the argon isotopes.
In order for this to work, a certain amount of \ce{^{36}Ar} is introduced into the tank as a carrier.
Through a sophisticated system, the argon is purified and finally, its \ce{^{37}Ar} content is measured in a proportional counter.
By counting the number of decaying argon isotopes and extrapolating using its half-life of \num{35} days, it is possible to calculate the number of neutrino interactions during the exposure.

A proportional counter is a container with two electrodes (usually a cylinder with a wire in its centre) filled with a counting medium (usually gaseous).
Incident charged particles ionise the counting medium---neutral particles can be detected if they first produce charged particles via interaction with matter in or surrounding the detector.
If an electric field is applied to the electrodes, the produced electron-ion pairs are separated and drift towards the corresponding electrode.
Reading out the current on the electrodes, one can measure the amount of ionisation produced inside the detector.
Usually, the anode is read out because the drift velocity of electrons in an electric field is much higher than the one of ions.
In this regime, the detector is in fact an ionisation counter rather than a proportional counter.
The problem is that the charge produced by the ionisation is very low and the current detector needs to be very sensitive.
By increasing the voltage across the electrodes, the sensitivity can be improved.
If the field inside the counter is above a certain threshold, the drifting ionisation electrons become energetic enough to ionise the counting medium themselves and thus, start an avalanche and produce more charge.
In the appropriate voltage range, the produced charge is still proportional to the primary ionisation charge, thus the name proportional counter.
The voltage can be raised further to enter the Geiger regime.
Avalanches will now produce UV photons in addition to the ionisation.
These UV photons travel independently of the electric field and can start new avalanches via the photoelectric effect.
The process can only be stopped by quenching the discharge either electrically (temporary voltage reduction) or chemically (quenching additive).

While the Homestake experiment provided a clean way of counting $\nu_e$ interactions, it provided no information on the timing, direction and kinematics of the interaction.
Only a lower energy threshold is given by the fact, that the neutrino needs to have enough energy to dissociate the chlorine atom from the tetrachloroethylene molecule.
Due to this, it was not possible to tell which reaction chain in the sun, the detected neutrinos originated from.
Furthermore, care needs to be taken for a very good understanding of all background processes that can produce \ce{^{37}Ar} or its signature in the counting tube.
Finally, this experiment was only capable of detecting $\nu_e$ which proved to be crucial in the solution to the solar neutrino problem; oscillation.

In 1988, the Kamioka Nucleon Decay Experiment (KamiokaNDE) and the Irvine-Michigan-Brookhaven (IMB) detector found a similar deficiency in atmospheric neutrinos which actually were a background for the original experiments looking for proton decays. %TODO: find source for IMB or remove
Atmospheric neutrinos are produced in a similar fashion than Lederman et al.\ did in their muon neutrino beam experiment.
Cosmic rays strike the Earth atmosphere and produce secondary particles many of which are pions, in turn, decaying according to Equation~\eqref{eq:nu-detection_pion-decay}.
Thus, atmospheric neutrinos are mainly $\nu_{\mu}/\overline{\nu}_{\mu}$.
However, the muon neutrino flux measured by KamiokaNDE was only \SI{59+-7}{\percent} of the one predicted by Monte Carlo simulations.~\cite{kamiokandeAtmos}
After an upgrade (Kamiokande II), the collaboration furthermore confirmed the solar neutrino problem discovered by the Homestake experiment.~\cite{kamiokandeSolar}
The detector was a \SI{3000}{\tonne} water tank equipped with \num{1000} PMTs to detect Cherenkov radiation produced by incoming charged particles.

Upon passage of a charge particle, the atoms of the medium become electric dipoles by means of polarisation.
If the velocity of the incident particle $v$ is greater than the the speed of light inside the medium $\frac{c}{n}$, defined by the refractive index $n$, this polarisation is not symmetric anymore, resulting in a non-vanishing dipole moment.
A characterisitic cone-shaped radiation in the direction of the particle is the result.
The half opening angle of the cone is given by
\begin{IEEEeqnarray}{rCl}
	\cos(\theta_{\m{c}}) & = & \frac{c}{n \qty(\lambda) v}
\end{IEEEeqnarray}
and the radiation spectrum is
\begin{IEEEeqnarray}{rCl}
	\dv{N}{x} & = & 2 \pi \alpha z ^ 2 \int_{\lambda_1}^{\lambda_2} \qty(\frac{\sin(\theta_{\m{c}} \qty(\lambda))}{\lambda}) ^ 2 \dd{\lambda}
\end{IEEEeqnarray}
with the number of Cherenkov photons $N$, path length $x$, fine-structure constant $\alpha$, and electric charge of the particle $z$.
By recording the ring produced by this cone with light detectors, it is possible to determine the timing, direction, momentum and type of the incident charged particle within certain restrictions.
Often employed detection media include water and oil while the photodetectors are usually PMTs.~\cite{grupen}

The charged particles detectable by a Cherenkov detector can be produced by neutrinos in multiple ways.
Here, only the two most important processes shall be introduced, a more detailed description will be given in Section~\ref{sec:nu_detection_interactions}.
Analogously to Equation~\eqref{eq:nu-detection_reinesCowan}, neutrinos of all three flavours can interact with nucleons according to
\begin{IEEEeqnarray}{rCl}
	\nu_l + n &				\rightarrow &	l^- + p \\
	\overline{\nu}_l + p &	\rightarrow &	l^+ + n
\end{IEEEeqnarray}
with $l = e,\mu,\tau$.
It should be noted however, that usually $\tau$ leptons are too short-lived to produce enough Cherenkov radiation to be detected.
A second interaction path of neutrinos with matter is the scattering off shell electrons whose recoil can be detected by Cherenkov detectors.
The kinematics of this process are independent of the neutrino flavour allowing for a detection of all three flavours.
This will play an important role in the resolution of the solar and atmospheric neutrino puzzle.

While registering timing and directionality in addition to being able to detect and distinguish $\nu_{\m{e}}$ and $\nu_{\m{\mu}}$ was a huge improvement over the radiochemical Homestake experiment, Cherenkov detectors still suffer from some difficiencies in particle identification.
One of them is that they can only detect charged particles with sufficient momentum to produce Cherenkov radiation rather than detecting the whole event topology.
The detector cannot distinguish between processes producing the same ring signature.
An important example is a $\pi^{\m{0}}$ produced by a $\nu_{\m{\mu}}$ which produces a signal in a Cherenkov detector very similar to the one of a $\nu_{\m{e}}$.
This is a crucial background for neutrino oscillation experiments.


\section{Neutrino interaction with matter\label{sec:nu_detection_interactions}}


\section{Final state detection\label{sec:nu_detection_fs}}
In order to be able to detect particles, they need to interact with a detection medium.
This section will describe the most important interaction of charged particles as well as neutral particles with matter.
A special focus is laid on charged interactions as these are the most important ones for \lartpc s.
As a measure of the interaction strength, the energy loss per distance or stopping power $\dv{E}{x}$ is used.
Where not otherwise mentioned, this section is based on~\cite{grupen}.

The main interaction of charged particles with matter happens on atomic electrons.
That is why for most of these interactions, one needs to treat the interaction of electrons separately.
For all other charged particles, the stopping power is described by the Bethe-Bloch formula
\begin{IEEEeqnarray}{rCl}
	- \frac{1}{\rho} \dv{E}{x} & = &
	4 \pi N_{\m{A}} r_{\m{e}} ^ 2 m_{\m{e}} c ^ 2 z ^ 2 \frac{Z}{A} \frac{1}{\beta ^ 2}
	\qty[\ln(\frac{2 m_{\m{e}} c ^ 2 \gamma ^ 2 \beta ^ 2}{I}) - \beta ^ 2 - \frac{\delta}{2}] \m{,}
	\label{eq:bethe-bloch}
\end{IEEEeqnarray}
where
\begin{itemize}
	\item[$\rho$] is the density of the absorber material,
	\item[$N_{\m{A}}$] is Avogadro's number,
	\item[$r_{\m{e}}$] $= \frac{1}{4 \pi \varepsilon_{\m{0}}} \frac{\si{\elementarycharge} ^ 2}{m_{\m{e}} c ^ 2}$ is the classical electron radius using the permittivity of free space $\varepsilon_{\m{0}}$,
	\item[$m_{\m{e}}$] is the electron mass,
	\item[$z$] is the charge of the incident particle,
	\item[$Z$] is the atomic number of the absorber,
	\item[$A$] is the atomic weight of the absorber,
	\item[$\beta$] $= \frac{v}{c}$ with $v$ the velocity of the incident particle,
	\item[$\gamma$] $= \frac{E}{m_0 c ^ 2}$ with $E$ the energy and $m_0$ the rest mass of the incident particle,
	\item[$I$] is the mean excitation energy of the absorber material which can be approximated by
		\begin{IEEEeqnarray}{rCl}
			I = 16 Z ^ {0.9} \si{\electronvolt} \quad \m{for} \quad Z > 1 \m{,}
		\end{IEEEeqnarray}
	\item[$\delta$] is a parameter describing the screening of the extended transverse electric field of relativistic incident particles by the charge density of the atomic electrons of the absorber.
\end{itemize}
Equation~\eqref{eq:bethe-bloch} describes the stopping power of particles with $m_0 \gg m_{\m{e}}$ by ionisation and excitation of the atoms in the absorber material.
As the stopping power is proportional to the electron density and thus to the mass density of the absorber material, it is often divided by the latter.
Thus, Equation~\eqref{eq:bethe-bloch} actually gives the so called mass stopping power.
The only remaining dependence on the absorber material is $\frac{Z}{A}$ which is $\approx 0.5$ for most light materials, and the mean excitation energy which only contributes logarithmically.
\begin{figure}[htbp]
	\includegraphics[width=\textwidth]{bethe-bloch}
	\caption{Bethe-Bloch stopping power of $\mu$ in \si{Fe}\label{fig:bethe-bloch}}
\end{figure}
Figure~\ref{fig:bethe-bloch} shows the mass stopping power of $\mu$ in \si{Fe} neglecting the $\frac{\delta}{2}$ term.
As can be seen, there is a broad minimum which is characteristic of the Bethe-Bloch formula.
Particles in this momentum range are called minimum ionising particles (MIPs).
They are important for detectors because this energy loss is a measure for the required energy resolution of a detector.
As mentioned above, the mass stopping power only loosely depends on the absorber material and therefore, its minimum is
\begin{IEEEeqnarray}{rCl}
	\eval{- \frac{1}{\rho} \dv{E}{x}}_{\m{min}} & \approx & \SI{2}{\mega\electronvolt\centi\meter\squared\per\gram}
\end{IEEEeqnarray}
for singly charged incident particles on most (light) absorbers.
To the left of the minimum is the \emph{Bragg peak} which is especially important for radiation therapy with heavy charged particles (e.g.\ protons).
The Bragg peak falls off with a strong $\frac{1}{\beta ^ 2}$ dependence.
After the minimum, the stopping power rises again with a logarithmic dependence on $\beta$ and the mean excitation energy of the absorber $I$.
The reason for this so called \emph{logarithmic rise} is the extension of the transverse electric field of the incident particle in the relativistic regime.
Due to increasing shielding of the transverse electric field by the shell electrons of the absorber materials, taken into account by the $\frac{\delta}{2}$ term, the rise is only asymptotic.
For electrons and positrons, Equation~\eqref{eq:bethe-bloch} does not hold because their mass is equal to the mass of the atomic electrons of the absorber.
The stopping power changes further for electrons because the incident particle cannot be distinguished form its collision partner in that case.
On the other hand, a positron will be annihilated upon stop by an electron which needs to be taken into account as well.
The equivalent of Equation~\eqref{eq:bethe-bloch} for $e^{\pm}$ can be found in~\cite{grupen}.

At high velocities, further effects come into play.
\emph{Bremsstrahlung} describes the radiation energy loss of a fast charged particle in the Coulomb field of the absorber nuclei.
It can be described by
\begin{IEEEeqnarray}{rCl}
	- \frac{1}{\rho}\dv{E}{x} & = & \frac{E}{X_{\m{0}}}
	\label{eq:bremsstrahlung}
\end{IEEEeqnarray}
where
\begin{IEEEeqnarray}{rCl}
	X_{\m{0}} & = & \frac{A}{4 \alpha N_A Z \qty(Z + 1) \qty(\frac{1}{4 \pi \varepsilon_{\m{0}}} \frac{\si{\elementarycharge} ^ 2}{m c ^ 2}) ^ 2 \ln(183 Z ^ {- \frac{1}{3}})}
	\label{eq:radiationlength}
\end{IEEEeqnarray}
is the \emph{radiation length} of the absorber material using
\begin{itemize}
	\item[$\alpha$] $\approx \frac{1}{137}$ the fine-structure constant and
	\item[$m$] the mass of the incident particle.
\end{itemize}
Again, the energy loss is proportional to the density of the absorber and for convenience, divided by the latter.
Bremsstrahlung is emitted in interactions of the incident particle with the absorber nuclei ($\propto Z ^ 2$) as well as with the atomic electrons of the absorber ($\propto Z$).
By neglecting the latter, one obtains the important relation
\begin{IEEEeqnarray}{rCl}
	X_0 ^ {- 1} & \propto & Z ^ 2
\end{IEEEeqnarray}
as opposed to the $\propto Z$ dependence of the Bethe-Bloch formula.
Equation~\eqref{eq:bremsstrahlung} also holds for electrons as long as $E \gg \frac{m_{\m{e}} c ^ 2}{\alpha Z ^ {\frac{1}{3}}}$.
Furthermore, looking at the dependence on the mass of the incident particle, one finds
\begin{IEEEeqnarray}{rCl}
	X_0 & \propto & m ^ 2
\end{IEEEeqnarray}
using Equation~\eqref{eq:radiationlength}.
Therefore, the radiation length of an absorber material is usually given for electrons and the relation
\begin{IEEEeqnarray}{rCl}
	X_0 & = & X_0^{\m{e}} \frac{m ^ 2}{m_{\m{e}} ^ 2}
\end{IEEEeqnarray}
can be used to get the radiation length for any charged particle of mass $m$.
Radiation losses play a significant role only at energies much higher than the energy of MIPs.
Using Equations~\eqref{eq:bethe-bloch} and~\eqref{eq:bremsstrahlung}, one can define a \emph{critical energy} $E_{\m{c}}$ by
\begin{IEEEeqnarray}{rCl}
	\eval{\dv{E}{x}_{\m{ion}}}_{E_{\m{c}}} & = & \eval{\dv{E}{x}_{\m{brems}}}_{E_{\m{c}}}
\end{IEEEeqnarray}
at which radiation losses take over from ionisation losses.
Similar to the radiation length, the critical energy is proportional to $m ^ 2$.
Thus, it is most important for electrons while for other particles it becomes significant only at very high energies.
If we take the example of an iron absorber again for instance, we get $E_c^{\m{e}} = \SI{20.7}{\mega\electronvolt}$ and $E_c^{\mu} = \SI{890}{\giga\electronvolt}$.

At high energies, there are additional types of radiation loss taking place, for instance direct electron-pair production and photonuclear interactions.
They shall not be described here.
Instead, only their $\propto E$ relation similar to bremsstrahlung losses shall be mentioned.
A description of those effects can be found in~\cite{grupen}.

Concerning the interactions of charged particles with matter, there is one important note regarding detectors.
While charge produced in interactions (i.e.\ ionisation) can be detected directly, light (i.e.\ excitation photons and photon radiation) first needs to be converted to charge to be detected.
This conversion from light to electric charge is the topic of the next section.

Photons can interact with matter in different ways.
In particular, they can be converted to charge which can be detected.
The three most important interactions converting photons to charge shall be outlined in this section.
All of them have in common that they attenuate photon beams exponentially according to
\begin{IEEEeqnarray}{rCl}
	I & = & I_0 \m{e} ^ {- \mu x}
\end{IEEEeqnarray}
where $I_0$ and $I$ is the intensity before and after passing the absorber, respectively.
The thickness of the absorber is given by $x$ and
\begin{IEEEeqnarray}{rCl}
	\mu & = & \frac{N_A}{A} \sum_i \sigma_i
	\label{eq:mass_att_coeff}
\end{IEEEeqnarray}
is the \emph{mass attenuation coefficient} defined by the sum of the cross sections $\sigma_i$ of the different interaction processes.

At low energies (ionisation energy $\le E_{\gamma} \le \SI{100}{\kilo\electronvolt}$), photons primarily undergo conversion to charge by the \emph{photoelectric effect}.
The photon is absorbed by an atom of the absorber which in turn is ionised and thus ejects one of its shell electrons.
The cross section is given by
\begin{IEEEeqnarray}{rCl}
	\sigma_{\m{photo}} = \qty(\frac{32}{\epsilon ^ 7}) ^ \frac{1}{2} \alpha ^ 4 Z ^ 5 \sigma_{\m{Th}}^{\m{e}}
\end{IEEEeqnarray}
where
\begin{itemize}
	\item[$\epsilon$] $= \frac{E_{\m{\gamma}}}{m_{\m{e}} c ^ 2}$ is the reduced photon energy and
	\item[$\sigma_{\m{Th}}^{\m{e}}$] $= \frac{8}{3} \pi r_{\m{e}} ^ 2 = \SI{6.65e-25}{\centi\meter\squared}$ is the \emph{Thomson cross section} for elastic scattering of photons on electrons.
\end{itemize}

For energies $\approx \SI{1}{\mega\electronvolt}$, \emph{Compton scattering} dominates the interaction of photons with matter.
Thereby, the photon is not absorbed by the atom but just scatters off one of its shell electrons with the cross section
\begin{IEEEeqnarray}{rCl}
	\sigma_{\m{c}} & = & 2 \pi r_{\m{e}} ^ 2 Z \left\{\qty[\frac{1 + \epsilon}{\epsilon ^ 2}] \qty[\frac{2 \qty(1 + \epsilon)}{1 + 2 \epsilon} - \frac{1}{\epsilon} \ln(1 + 2 \epsilon)]\right.\\
	& & \left. + \frac{1}{2 \epsilon} \ln(1 + 2 \epsilon) - \frac{1 + 3 \epsilon}{\qty(1 + 2 \epsilon) ^ 2}\right\}
\end{IEEEeqnarray}
obtained from the Klein-Nishina formula.
As only part of the photon's energy is absorbed while the rest is scattered, it makes sense to divide this cross section into a scattering cross section
\begin{IEEEeqnarray}{rCl}
	\sigma_{\m{cs}} & = & \frac{E_{\gamma}'}{E_{\gamma}}
\end{IEEEeqnarray}
and an absorption cross section
\begin{IEEEeqnarray}{rCl}
	\sigma_{\m{ca}} & = & \sigma_{\m{c}} - \sigma_{\m{cs}}
\end{IEEEeqnarray}
where $E_{\gamma}$ and $E_{\gamma}'$ is the energy of the photon before and after scattering, respectively.

At $E_{\gamma} \ge 2 m_{\m{e}} c ^ 2$, photons are capable of producing pairs of $e^+ e^-$.
Because of momentum conservation, this process can only happen in the coulomb field of a so called spectator particle.
As pair-production in the field of an electron is strongly suppressed, the spectator is usually a nucleus of the absorber material.
Therefore, the cross section of pair-production depends on the shielding of the coulomb field by the shell electrons and thus on the proximity to the nucleus.
Eventually, this results in an energy dependence.
For $1 \ll \epsilon < \frac{1}{\alpha Z ^ {\frac{1}{3}}}$, the cross section is given by
\begin{IEEEeqnarray}{rCl}
	\sigma_{\m{pair}} & = & 4 \alpha r_{\m{e}} ^ 2 Z ^ 2 \qty(\frac{7}{9} \ln 2 \epsilon - \frac{109}{54})
\end{IEEEeqnarray}
and for $\epsilon \gg \frac{1}{\alpha Z ^ {\frac{1}{3}}}$, the cross section is
\begin{IEEEeqnarray}{rCl}
	\sigma_{\m{pair}} & = & 4 \alpha r_{\m{e}} ^ 2 Z ^ 2 \qty[\frac{7}{9} \ln(\frac{183}{Z ^ {\frac{1}{3}}}) - \frac{1}{54}] \m{.}
\end{IEEEeqnarray}

As mentioned above, for Compton scattering, two different cross sections are defined, one for the scattered energy ($\sigma_{\m{cs}}$) and one for the absorbed one ($\sigma_{\m{ca}}$).
Consequentially, there are also different definitions of the coefficient $\mu$ in Equation~\eqref{eq:mass_att_coeff}.
Replacing $\sigma_{\m{c}}$ by $\sigma_{\m{cs}}$, $\mu_{\m{s}}$ is the \emph{mass attenuation coefficient} and similarly, using $\sigma_{\m{ca}}$, $\mu_{\m{s}}$ is the \emph{mass absorption coefficient}.
While $\mu$ is more precisely called the \emph{total mass attenuation coefficient}.

Hardons.