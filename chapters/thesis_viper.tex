\chapter{Results from the first measurements with the \AC\ demonstrator\label{chap:viper}}


\section{Experimental setup\label{sec:viper_setup}}
The experiments were performed in a bath cryostat with a diameter of \SI{50}{\centi\metre} and a height of \SI{110}{\centi\metre} which gives and inner volume of $\approx \SI{200}{\litre}$ of liquid argon.
The cryostat is equipped with a recirculation pump for continous purification during operation.
A purity of $\approx \SI{1}{ppb}$ can be reached
Before filling, the vessel was evacuated to $\approx \SI{1e-3}{\milli\bar}$ using a roots vacuum pump, then purged with argon gas and evacuated again.
The high voltage feedthrough is made from PET and is the same used for the breakdown experiments described in Section~\ref{chap:hv} and~\cite{breakdown_14, breakdown_16}.

The newly designed TPC has a length of \SI{63}{\centi\metre} and a diameter of \SI{10}{\centi\metre}. %TODO: Check dimensions!
For the field generation, \num{30} aluminium rings are stacked with a pitch of \SI{20}{\milli\metre}.
In between the field shaping rings, acrylic rings are placed for light collection.
A more detailed description of the light collection system can be found in Section~\ref{sec:rd-dune-nd_light}.
To allow proper convection of the argon, every other acrylic ring is reduced to four small pieces attached to four support struts made from PAI.


\section{3D reconstruction\label{sec:viper_reco}}
To characterise the pixel readout, together with the Bern group I will implement a truly 3D reconstruction in the LArSoft framework.
This allows the comparison of the pixel readout performance with contemporary projected wire readouts.
To have a direct comparison, we use 2D projections of the data taken with the pixels, thus eliminating any bias not related to projection.
This enables future experiments to reconstruct data from pixelated readouts.