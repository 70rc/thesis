\begin{abstract}

\glsreset{lartpc}
\glsreset{dune}
\glsreset{nd}
\glsreset{fd}
\glsreset{sm}
\glsreset{cp}

The \gls{sm} of particle physics has proven to be remarkably precise in its predictions.
However, there are still some missing pieces like the intriguing nature of neutrinos.
In particular do their mass eigenstates mix with their flavour eigenstates resulting in oscillation for neutrino masses different from zero.
Neutrino mixing and oscillations have been extensively studied during the last few decades probing the parameters of the three flavour model.
Nevertheless, several unanswered questions remain, in particular the possible existence of a \gls{cp} violating phase in the mixing matrix and the ordering of the neutrino mass eigenstates.
The next-generation \dune{} is being built to answer them via a detailed study of long-baseline neutrino oscillations.
Achieving sensitivity to \gls{cp} violation and mass ordering will require a data sample of unprecedented size and precision.
Therefore, a high-intensity beam (\SI{2}{\mega\watt}) and massive detectors (\SI{40}{\kilo\tonne}) providing excellent tracking and calorimetry are required.
\glspl{lartpc} fulfil these requirements excellently and were therefore chosen as \glspl{fd}.
To bring systematic uncertainties down to the required level of a few percent, a \lartpc{} component in the \gls{nd} complex is necessary.
Even though, the high-rate \gls{nd} environment poses significant challenges to traditional \lartpc{} designs.
Their wire charge readout reduces the excellent \gls{3d} tracking capabilities of a \lartpc{} to a number of \gls{2d} projections.
Furthermore, \lartpc{}s are comparatively slow detectors due to the finite charge drift velocity ($\sim \SI{1}{\milli\metre\per\micro\second}$).
To overcome limitations a pixelated charge readout for \lartpc{}s was developed and successfully tested in the framework of this thesis.
A software framework was established to reconstruct recorded cosmic muon tracks employing a Kalman filter.
Pixelated charge readout systems represent the single largest advancement in the sensitivity of \lartpc{}s enabling true \gls{3d} tracking, thereby reducing event pile-up and improving background rejection as will be shown.
The large volumes required by the \dune{} \lartpc{}s result in longer drift distances and thus require higher drift voltages.
A method was developed to operate \lartpc{}s at electric fields an order of magnitude higher than before.
It was found however that it is reasonable to keep fields below \SI{40}{\kilo\volt\per\centi\metre} at all points in the detector to guarantee a safe operation.
The work presented here is aimed towards the development of a new fully-modular, pixelated \lartpc{} concept---\AC{}.
Splitting the detector volume into independent self-contained \glspl{tpc} sharing a common \lar{} bath reduces the required drift voltages to a handleable level.
A pixelated charge readout paired with new electronics will exploit true \gls{3d} tracking to cope with the expected event pile-up.
Using simulations it was found that the impact of pile-up on the reconstructed neutrino energy in \AC{} has a mean between \SIlist{2;3}{\percent}, and is below \SI{0.1}{\percent} for more than \SI{50}{\percent} of the neutrino events.
At the time of writing, \AC{} is the top candidate for the \lar{} component in the \dune{} \gls{nd} complex.

\end{abstract}

\clearpage