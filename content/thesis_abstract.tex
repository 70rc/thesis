\begin{abstract}

\glsreset{lartpc}
\glsreset{dune}
\glsreset{nd}
\glsreset{fd}

Neutrino mixing and oscillations have been extensively studied during the last few decades.
Nevertheless, several unanswered questions remain, in particular \gls{cp} violation in the lepton sector and the ordering of the neutrino masses.
A next-generation \dune{} is being built to answer them via observation of long-baseline neutrino oscillations.
\glspl{lartpc} were chosen as \glspl{fd}.
In addition, a \lartpc{} component in the \gls{nd} complex is necessary.
Even though, the high-rate \gls{nd} environment poses significant challenges to traditional \lartpc{} designs.
Their wire charge readout reduces the excellent \gls{3d} tracking capabilities of a \lartpc{} to a number of \gls{2d} projections.
In addition, \lartpc{}s are comparatively slow detectors due to the finite charge drift velocity ($\sim \SI{1}{\milli\metre\per\micro\second}$).
A number of improvements are presented in this thesis.
To overcome limitations, a pixelated charge readout for \lartpc{}s was developed and successfully tested.
A software framework was established to reconstruct recorded cosmic muon tracks employing a Kalman filter.
Pixelated charge readout systems represent the single largest advancement in the sensitivity of \lartpc{}s, enabling true \gls{3d} tracking, thereby reducing event pile-up and improving background rejection.
The large volumes required by the \dune{} \lartpc{}s result in longer drift distances and thus require higher drift voltages.
A method was developed to operate \lartpc{}s at electric fields an order of magnitude higher than before.
It was found however that it is reasonable to keep fields below \SI{40}{\kilo\volt\per\centi\metre} at all points in the detector to guarantee a safe operation.
The presented R\&D is aimed towards the development of a new fully-modular, pixelated \lartpc{} concept---\AC{}.
Splitting the detector volume into small self-contained \gls{tpc} sharing a common \lar{} bath, reduces the required drift voltages to a handleable level.
A pixelated charge readout paired with new electronics will exploit true \gls{3d} tracking to cope with the expected event pile-up.
It was found that the impact of pile-up on the reconstructed neutrino energy in \AC{} has a mean between \SIlist{2;3}{\percent}, and is below \SI{0.1}{\percent} for more than \SI{50}{\percent} of the neutrino events.
At the time of writing, \AC{} is the top candidate for the \lar{} component in the \dune{} \gls{nd} complex.

\end{abstract}

\clearpage