\begin{abstract}

\glsreset{lartpc}
\glsreset{dune}
\glsreset{nd}
\glsreset{fd}
\glsreset{sm}
\glsreset{cp}
\glsreset{arclight}

The \gls{sm} of particle physics has proven to be remarkably consistent in its explanation of experimental observations.
An exception is the intriguing nature of neutrinos.
Particularly, neutrino flavour eigenstates do not coincide with their mass eigenstates.
The flavour eigenstates are a mixture of the mass eigenstates resulting in oscillations for non-zero neutrino masses.
Neutrino mixing and oscillations have been extensively studied during the last few decades probing the parameters of the three flavour model.
Nevertheless, unanswered questions remain: the possible existence of a \gls{cp} violating phase in the mixing matrix and the ordering of the neutrino mass eigenstates.
The \dune{} is being built to answer these questions via a detailed study of long-baseline neutrino oscillations.
Like any beam experiment, \dune{} requires two detectors: one near the source to characterise the unoscillated beam, and one far away to measure the oscillations.
Achieving sensitivity to \gls{cp} violation and mass ordering will require a data sample of unprecedented size and precision.
A high-intensity beam (\SI{2}{\mega\watt}) and massive detectors (\SI{40}{\kilo\tonne} at the far site) are required.
The detectors need to provide excellent tracking and calorimetry.
\glspl{lartpc} were chosen as \glspl{fd} because they fulfil these requirements.
A \lartpc{} component is also necessary in the \gls{nd} complex to bring systematic uncertainties down to the required level of a few percent.
A drawback of \lartpc{}s is their comparatively low speed due to the finite charge drift velocity ($\sim \SI{1}{\milli\metre\per\micro\second}$).
Coupled with the high beam intensity this results in event rates of \num{0.2}~piled-up events per tonne in the \gls{nd}.
Such a rate poses significant challenges to traditional \lartpc{}s:
Their \gls{3d} tracking capabilities are limited by wire charge readouts providing only \gls{2d} projections.
To address this problem a pixelated charge readout was developed and successfully tested as part of this thesis.
This is the first time pixels were deployed in a single-phase \lartpc{} representing the single largest advancement in the sensitivity of \lartpc{}s---enabling true \gls{3d} tracking.
A software framework was established to reconstruct cosmic muon tracks recorded with the pixels.
Another problem with traditional \lartpc{}s is the large volume required by their monolithic design resulting in long drift distances.
Consequentially, high drift voltages are required.
Current \lartpc{}s are operating at the limit beyond which electric breakdowns readily occur.
This prompted world-leading studies of breakdowns in \lar{} including high-speed footage, current-voltage characteristics, and optical spectrometry.
A breakdown-mitigation method was developed which allows \lartpc{}s to operate at electric fields an order of magnitude higher than previously achieved.
It was found however that a safe and prolonged operation can be achieved more effectively by keeping fields below \SI{40}{\kilo\volt\per\centi\metre} at all points in the detector.
Therefore, high inactive clearance volumes are required for traditional monolithic \lartpc{}s.
Avoiding dead \lar{} volume intrinsically motivates a segmented \gls{tpc} design with lower cathode voltages.
The comprehensive conclusion of the \gls{hv} and charge readout studies is the development of a novel fully modular and pixelated \lartpc{} concept---\AC{}.
Splitting the detector volume into independent self-contained \glspl{tpc} sharing a common \lar{} bath reduces the required drift voltages to a manageable level and minimises inactive material.
\AC{} is incompatible with traditional \gls{pmt}-based light readouts occupying large volumes.
A novel cold \gls{sipm}-based light collection system utilised in the pixel demonstrator \gls{tpc} enabled the development of the compact \AL{}.
\AC{}'s pixelated charge readout will exploit true \gls{3d} tracking, thereby reducing event pile-up and improving background rejection.
Results of the pixel demonstration were used in simulations of the impact of pile-up for \AC{} in the \dune{} \gls{nd}.
The influence piled-up \Pgpz-induced \gls{em} showers have on neutrino energy reconstruction was investigated.
Misidentified neutrino energy in \AC{} is conservatively below \SI{0.1}{\percent} for more than \SI{50}{\percent} of the neutrino events, well within the \dune{} error budget.
The work described in this thesis has made \AC{} the top candidate for the \lar{} component in the \dune{} \gls{nd} complex.

\end{abstract}

\clearpage