\begin{abstract}

\glsreset{lartpc}
\glsreset{dune}
\glsreset{nd}
\glsreset{fd}

Neutrino mixing and oscillations have been extensively studied during the last few decades.
Nevertheless, several unanswered questions remain, in particular CP violation in the lepton sector and the ordering of the neutrino masses.
A next-generation \dune{} is being built to answer them via observation of long-baseline neutrino oscillations.
Achieving sensitivity to CP violation and mass ordering will require a data sample of unprecedented size and precision.
Therefore, a high-intensity beam (\SI{2}{\mega\watt}) and a massive detector (\SI{40}{\kilo\tonne}) providing excellent tracking and calorimetry are required.

\glspl{lartpc} fulfil these requirements excellently and were therefore chosen as \glspl{fd}.
To bring systematic uncertainties down to the required level of a few percent, a \lartpc{} component in the \gls{nd} complex is desirable.
The high-rate \gls{nd} environment poses significant challenges to traditional \lartpc{} designs.

\lartpc{}s have used projective wire readouts for charge detection since their conception in 1977.
However, a wire readout reduces the excellent 3D tracking capabilities of a \lartpc{} to a number of 2D projections.
In addition, \lartpc{}s are comparatively slow detectors due to the finite charge drift velocity (\SI{2}{\milli\metre\per\micro\second}).
Paired with the high beam intensity, this leads to several neutrino interactions piling up within a single readout cycle (\SI{0.2}{evt\per\tonne_{\lar}}).
Disentangling such an event pile-up only from 2D projections will be extremely challenging.
To overcome these limitations, a pixelated charge readout for \lartpc{} was developed and successfully tested.
A software framework was established to reconstruct recorded cosmic muon tracks employing a Kalman filter.
Pixelated charge readout systems represent the single largest advancement in the sensitivity of \lartpc{}s, enabling true 3D tracking, thereby reducing event pile-up and improving background rejection.

Due to their increased number of channels (roughly squared compared to wires), pixelated charge readouts give raise to the need for novel readout electronics.
Existing wire readout electronics were tested and found unsuitable for the new readout.
Together with the successful demonstration of the readout itself, this triggered the development of bespoke pixel electronics for the \dune{} \gls{nd}.

The large volumes required by the \dune{} \lartpc{}s result in longer drift distances and thus require higher drift voltages.
Recent studies have shown the dielectric strength of \lar{} to be much lower than predicted by earlier work.
This triggered an in-depth study of electric breakdowns including the recording of high-speed footage, current-voltage characteristics, and spectrometry.
As a result of these studies, a method was developed to increase the dielectric strength of \lar{} by an order of magnitude.
It was found however that it is safer to keep fields below \SI{40}{\kilo\volt\per\centi\metre} at all points in the detector to guarantee a safe operation.

The entirety of the R\&D effort lead to the development of a new fully-modular, pixelated \lartpc{} concept---\AC{}.
It is aimed to address the studied challenges.
Splitting the detector volume into small self-contained \gls{tpc} sharing a common \lar{} bath, reduces the required drift voltages to a handleable level.
A pixelated charge readout paired with bespoke electronics will exploit true 3D tracking to cope with the expected event pile-up.

The work is topped off with a study of the \gls{nd} event pile-up.
Simulated \Pgpz-induced electromagnetic showers were reconstructed by a rudimentary algorithm assuming unambiguous 3D information on charge deposition.
It was found that the impact of pile-up on the reconstructed neutrino energy has a mean between \SIlist{2;3}{\percent}, and is below \SI{0.1}{\percent} for more than \SI{50}{\percent} of the neutrino events.

At the time of writing, \AC{} is the top candidate for the \lar{} component in the \dune{} \gls{nd} complex.

\end{abstract}

\clearpage