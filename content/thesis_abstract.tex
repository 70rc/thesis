\begin{abstract}

\glsreset{lartpc}
\glsreset{dune}
\glsreset{nd}
\glsreset{fd}
\glsreset{sm}
\glsreset{cp}
\glsreset{arclight}

The \gls{sm} of particle physics has proven to be remarkably precise in its predictions with the exception of a few missing pieces like the intriguing nature of neutrinos.
Particularly neutrino flavour eigenstates do not coincide with their mass eigenstates; the flavour eigenstates are a mixture of the mass eigenstates resulting in oscillations for non-zero neutrino masses.
Neutrino mixing and oscillations have been extensively studied during the last few decades probing the parameters of the three flavour model.
Nevertheless, several unanswered questions remain: the possible existence of a \gls{cp} violating phase in the mixing matrix and the ordering of the neutrino mass eigenstates.
The next-generation \dune{} is being built to answer these questions via a detailed study of long-baseline neutrino oscillations.
Like any beam experiment, \dune{} requires two detectors: one near the source to characterise the unoscillated beam, and one far away to measure the oscillated beam.
Achieving sensitivity to \gls{cp} violation and mass ordering will require a data sample of unprecedented size and precision.
Therefore, a high-intensity beam (\SI{2}{\mega\watt}) and massive detectors (\SI{40}{\kilo\tonne} at the far site) providing excellent tracking and calorimetry are required.
\glspl{lartpc} fulfil these requirements excellently and were therefore chosen as \glspl{fd}.
To bring systematic uncertainties down to the required level of a few percent, a \lartpc{} component in the \gls{nd} complex is also necessary.
The high-rate \gls{nd} environment poses significant challenges to traditional \lartpc{} designs:
Their wire charge readout reduces the excellent \gls{3d} tracking capabilities of a \lartpc{} to a number of \gls{2d} projections.
Furthermore, \lartpc{}s are comparatively slow detectors due to the finite charge drift velocity ($\sim \SI{1}{\milli\metre\per\micro\second}$).
To overcome limitations a pixelated charge readout for \lartpc{}s was developed and successfully tested in the framework of this thesis.
This was the first time pixels were deployed in a single-phase \lartpc{} representing the single largest advancement in the sensitivity of \lartpc{}s---enabling true \gls{3d} tracking.
The development and characterisation of the pixels is described, along with a software framework that was established to reconstruct recorded cosmic muon tracks employing a Kalman filter.
The pixel demonstrator \gls{tpc} also featured a novel cold \gls{sipm}-based light collection system, which lead to the development of the \AL{}, the characterisation of which is also described.
The large volumes required by the \dune{} \lartpc{}s result in longer drift distances and thus require higher drift voltages than contemporary detectors.
World-leading studies of \gls{hv} breakdowns in \lar{} including high-speed footage, current-voltage characteristics, and optical spectrometry are presented. 
A breakdown-mitigation method was developed which allows \lartpc{}s to operate at electric fields an order of magnitude higher than previously achieved.
To guarantee a safe and prolonged operation it was found that it is more effective to keep fields below \SI{40}{\kilo\volt\per\centi\metre} at all points in the detector.
The comprehensive conclusion of the \gls{hv} and charge readout studies is the development of a new fully modular and pixelated \lartpc{} concept---\AC{}.
Splitting the detector volume into independent self-contained \glspl{tpc} sharing a common \lar{} bath reduces the required drift voltages to a manageable level and minimises inactive material.
A pixelated charge readout paired with new electronics will exploit true \gls{3d} tracking, thereby reducing event pile-up and improving background rejection, as will be shown.
Results of the pixel feasibility study were used to simulate the impact of pile-up for \AC{} in the \dune{} \gls{nd}.
The influence piled-up \Pgpz-induced \gls{em} showers have on neutrino energy reconstruction was investigated.
Misidentified neutrino energy in \AC{} is conservatively below \SI{0.1}{\percent} for more than \SI{50}{\percent} of the neutrino events, well within the \dune{} error budget.
The work described in this thesis has made \AC{} the top candidate for the \lar{} component in the \dune{} \gls{nd} complex.

\end{abstract}

\clearpage