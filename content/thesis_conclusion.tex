\chapter{Conclusion}
\label{chap:conclusion}

\dune{} is a long-baseline neutrino oscillation experiment aiming to discover \gls{cp} violation in the lepton sector and determine the neutrino mass ordering.
\lartpc{}s will be deployed in the \gls{fd} complex due to their excellent tracking and calorimetric capabilities.
A \lartpc{} component is also required in the \gls{nd} complex to bring beam-related systematic uncertainties below the required \SI{2}{\percent}.
The \gls{nd} environment will be very challenging due to the slow readout ($\approx \SI{0.5}{\milli\second\per\metre}$) of \lartpc{}s compared to the beam spill duration (\SI{10}{\micro\second}).
The high beam intensity will therefore lead to \num{0.2}~neutrino events per tonne of argon and beam spill.
In this thesis most of the relevant challenges for \lartpc{}s in future high-multiplicity environments were studied alongside potential solutions, namely the dielectric strength of \lar{}, new charge and light readout methods, as well as the required next-generation charge readout electronics.

The \AT{} detector demonstrator built at \gls{help} found the dielectric strength of \lar{} to be much lower than the predicted $\approx \SI{1}{\mega\volt\per\centi\metre}$.
This led to a systematic study of dielectric breakdowns in \lar{}.
In particular, we found that the dielectric strength is dependent on absolute dimensions.
I recorded and analysed high-speed footage, current-voltage characteristics, and optical spectrometry of breakdowns.
A conclusive theory of dielectric breakdowns in \lar{} at the centimetre scale was developed~\cite{breakdown_16}.
The phenomenon is governed by three distinct phases: field emission, streamer, and breakdown.
Understanding the process enabled the development of a technique to mitigate breakdowns~\cite{latex}.
However, this solution proved to be unreliable.
Only keeping fields below \SI{40}{\kilo\volt\per\centi\metre} everywhere in the detector guarantees a safe operation.
This can either be reached by decreasing cathode voltages or increasing uninstrumented clearance volumes around \gls{hv} components.
Avoiding additional dead \lar{} volume intrinsically motivates a segmented \gls{tpc} design with lower cathode voltages.

Classical wire plane readouts of \lartpc{}s have significant drawbacks.
Besides their mechanical fragility, they cripple the excellent \gls{3d} tracking capabilities of a \gls{tpc} by reducing it to multiple \gls{2d} projections.
This is highly problematic in high-multiplicity environments such as the \dune{} \gls{nd} due to the complex event topologies resulting from event pile-up.
In a preliminary study I showed that the mechanical challenges met by wire plane charge readouts can be alleviated by replacing the wires with copper tracks printed on a thin Kapton layer.
However, this does not solve the inherent ambiguities caused by wires.
A true \gls{2d} readout in form of pixels is needed instead.

Realising a pixelated \lartpc{} is complicated by the high number of channels.
Cold digitisation can help by aggregating many pixels on a single high-speed digital link, reducing the number of required cable feedthroughs out of the cryostat.
I evaluated the cold digitisers foreseen for the \dune{} \gls{fd} and found them to be unsuitable for a pixelated \gls{nd}.
Being optimised for wire readouts their power dissipation is much too high given the required number of channels.
With no suitable cold electronics at hand I implemented a form of analogue multiplexing to demonstrate a pixelated \lartpc{} at the price of introducing some ambiguities in the \gls{3d} spatial information.
Like this it was possible to use the existing charge readout electronics from \AT{}.
The successful demonstration of pixels provides the basis for the charge readout in \AC{}.

I designed and built a new prototype \gls{tpc} to test the pixelated charge readout scheme.
I extended the \AT{} readout electronics by a differential warm signal path and reduced the parasitic capacitances in the pixel readout \gls{pcb}.
With this improvements an \gls{snr} of \num{14} was reached, proving pixels feasible for operation in real physics experiments.
Together with the \lar{} group of \gls{help} I successfully recorded several thousand cosmic muon tracks.
These results triggered the development of bespoke cold pixel electronics, \larpix{}, by \gls{lbnl} aimed to eventually enable an ambiguity-free pixelated charge readout in \AC{}.
\larpix{} uses a smart zero suppression scheme to meet the stringent power dissipation requirements of a pixelated charge readout.

I developed a new software framework to reconstruct the cosmic muon tracks recorded with the pixel demonstrator.
The hit finder had to be written from scratch because all existing \lartpc{} reconstruction frameworks are optimised for wire readouts.
A \gls{pca} was employed to solve the ambiguities stemming from the analogue multiplexing.
Finally, the unambiguous \gls{3d} measurements were fed to \gls{genfit}, an existing generic track-fitting toolkit based on a Kalman filter.
Therewith, I obtained fully reconstructed cosmic muon tracks, illustrating the advantages of the \AC{} approach over existing schemes.
This software framework serves as a starting point for future efforts on the reconstruction of true \gls{3d} spatial information recorded with \AC{}.
Both, pixel demonstration and \gls{3d} event reconstruction, have been presented at conferences~\cite{pixel_proceedings} by me and published in a paper~\cite{pixel_paper} of which I am corresponding author.

The pixel demonstrator \gls{tpc} was also used to test the operation of \glspl{sipm} in \lar{} for the light trigger system.
Based on the findings \gls{help} developed \AL{}~\cite{arclight}, a light trap maximising the area coverage of \glspl{sipm} while minimising the occupied volume.
It provides a compact light readout for \AC{} which cannot use a classic \gls{pmt}-based light readout occupying large volumes.
I contributed to the testing and characterisation of \AL{}.

Scaled up versions of both, \AL{} and the pixelated charge readout, were successfully tested in the \pixlar{} test beam experiment at \gls{fail}.
These results pave the way for an application of both technologies in \AC{}.

Finally, I performed an event pile-up study using simulated \Pgpz decay photons to demonstrate the ability of a pixelated \lartpc{} to cope with the high event rates expected in the \dune{} \gls{nd}.
At \dune{} energies such photons produce \gls{em} showers consisting of a plethora of small disconnected charge depositions in the detector.
Correctly associating these to the right neutrino event is one of the most difficult reconstruction tasks.
At the same time, failure to reconstruct them properly significantly distorts the reconstructed neutrino energy spectrum.
Based on the results from the pixel demonstrator I assumed unambiguous \gls{3d} position information for the charge depositions.
I employed a simple cone-based algorithm to associate the charge to the corresponding photon.
The mean deposited energy missed by the algorithm was found to be less than \SI{3}{\percent}.
More importantly the pile-up-related misidentification of energy depositions from other events was found to have a mean of \SIrange{2}{3}{\percent}.
For more than \SI{50}{\percent} of the neutrino events this error is even smaller than \SI{0.1}{\percent}.
If a way is found to flag the other \SI{50}{\percent} as piled-up events, a sample of neutrino events almost free of pile-up can be generated.
In comparison, the \gls{fd} is required to have an energy resolution for stopping hadrons below \SI{10}{\percent} and an electron energy resolution of $\SI{1}{\percent} \oplus \SI{15}{\percent} \times \sqrt{\frac{\SI{1}{\mega\electronvolt}}{E}}$.
Therefore, a pixelated \AC{} providing unambiguous \gls{3d} tracking information will be capable of handling the high rates expected in the \dune{} \gls{nd} environment without significant contributions to the error budget.

The combination of results from all of this work builds the groundwork for the \AC{}, a novel fully modular \lartpc{} concept, addressing the most important challenges of future neutrino detectors, in particular the \dune{} \gls{nd}.
High cathode voltages are prevented by splitting the detector into several small, self-contained \glspl{tpc} requiring only a moderate \SI{50}{\kilo\volt} cathode voltage.
A pixelated charge readout enables the true \gls{3d} tracking required to cope with the high event rates resulting from the high-intensity neutrino beam.
The \AL{} readout minimises the occupied dead volume inside the modules resulting in a similar performance to a monolithic detector.
At the same time, the scintillation light is contained within each module, simplifying association to the correct ionisation signals.

With the most important key technologies for \AC{} tested the path is clear for the first modules in the \num{2 x 2} module prototype at \gls{help}, which will test the unification of all the pieces provided by this work.
Cosmic ray events will be recorded and analysed with the developed reconstruction framework to characterise the physics performance of \AC{}.
After successful cosmic tests, beam tests will follow at either CERN or \gls{fail}.
The improvements I made allow \lartpc{}s to operate in high-multiplicity environments and led to \AC{} being the baseline \lar{} component of the \dune{} \gls{nd} complex.

Several aspects of the presented work can be continued in the future to further improve \lartpc{} technology.
Finding a more reliable coating material than latex for \gls{hv} components could enable large monolithic detectors.
A continuous resistive field cage has the potential to provide a highly uniform electric field with a simple mechanical structure.
Recent tests indicate that the power dissipation of \larpix{} is low enough to make a pixelated \dune{} \gls{fd} conceivable.
Even though not deemed a requirement as of today, this would simplify event reconstruction tremendously and improve sensitivities accordingly.
Finally, the modular \AC{} concept can easily be adapted to other experiments requiring a high-density, high-precision tracker and calorimeter.