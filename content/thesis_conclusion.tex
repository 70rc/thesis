\chapter{Conclusion}
\label{chap:conclusion}

In this thesis, all relevant challenges for \lartpc{}s in high-multiplicity environments were studied, namely the dielectric strength of \lar{}, new charge and light readout methods, as well as the required next-generation charge readout electronics.
After the \AT{} experiment~\cite{AT} at \gls{help} found the dielectric strength of \lar{} to be much lower than the $\approx \SI{1}{\mega\volt\per\centi\metre}$ predicted by studies in the fifties, a systematic study of dielectric breakdowns in \lar{} was undertaken.
In particular, it was found that the dielectric strength is dependent on absolute dimensions~\cite{breakdown_14}.
I recorded high-speed footage, current-voltage characteristics, and spectrometry of breakdowns.
From this, a conclusive theory of dielectric breakdowns in \lar{} at the centimetre scale was developed~\cite{breakdown_16}.
The phenomenon is governed by three distinct phases: field emission, streamer, and breakdown.
Understanding the process enabled the development of a technique to mitigate the breakdowns~\cite{latex}.
However, this solution proved to be unreliable.
Only keeping fields below \SI{40}{\kilo\volt\per\centi\metre} everywhere in the detector guarantees a safe operation.

In a preliminary study, I showed that the mechanical challenges met by wire plane charge readouts can be alleviated by replacing the wires with copper tracks printed on a thin Kapton layer.
However, this does not solve the inherent ambiguities caused by wires.
Instead, a true \gls{2d} readout in form of pixels is needed.
To address the high number of channels required by a pixelated charge readout, I evaluated the cold digitisers foreseen for the \dune{} \gls{fd} and found them unsuitable for a pixelated \gls{nd}.
Being optimised for wire readouts, their power dissipation is much too high given the required number of channels.
With no suitable cold electronics at hand, I implemented a form of analogue multiplexing to demonstrate a pixelated \lartpc{}.
Like this, it was possible to use the existing charge readout electronics from \AT{}.

I helped design and build a further prototype \gls{tpc} to test the pixelated charge readout scheme.
Besides, it was also used to test the operation of \glspl{sipm} in \lar{} for the light trigger system.
Together with the \lar{} group of \gls{help}, I successfully recorded several thousand cosmic muon tracks.
The first measurement campaign still had some noise on the charge readout.
With the help of the electronics workshop, I extended the \AT{} readout electronics by a differential warm signal path, and reduced the parasitic capacitances in the pixel readout \gls{pcb}.
In the second run, a \gls{snr} of \num{14} on the pixel channels could be reached, proving a sufficient performance of pixels for operation in a real physics experiment.
These results triggered the development of bespoke cold pixel electronics, \larpix{}, by \gls{lbnl}, aimed to eventually enable an ambiguity-free pixelated \lartpc{} charge readout.

To reconstruct the recorded cosmic muon tracks, I developed a new software framework.
The hit finder had to be written from scratch because all existing \lartpc{} reconstruction frameworks are optimised for wire readouts.
A \gls{pca} was employed to solve the ambiguities stemming from the analogue multiplexing.
Finally, the unambiguous \gls{3d} measurements were fed to \gls{genfit}, an existing generic track-fitting toolkit based on a Kalman filter.
Therewith, I obtained fully reconstructed cosmic muon tracks.

Finally, I performed an event pile-up study using simulated \Pgpz decay photons to demonstrate the ability of a pixelated \lartpc{} to cope with the high event rates expected in the \dune{} \gls{nd}.
Based on the results from the pixel prototype test, I assumed unambiguous \gls{3d} position information for the charge depositions.
I employed a simple cone-based algorithm to associate the charge to the corresponding photon.
The mean deposited energy missed by the algorithm was found to be less than \SI{3}{\percent}.
More importantly, the pile-up-related misidentification of energy depositions from other events was found to have a mean of \SIrange{2}{3}{\percent}.
For more than \SI{50}{\percent} of the neutrino events, this error is even smaller than \SI{0.1}{\percent}.
If a way is found to flag the other \SI{50}{\percent} as piled up during event reconstruction, a sample of neutrino events almost free of pile-up can be generated.
In comparison, the \gls{fd} is required to have an energy resolution for stopping hadrons below \SI{10}{\percent} and an electron energy scale uncertainty of about \SI{5}{\percent}.
Therefore, a pixelated \lartpc{} providing unambiguous \gls{3d} tracking information will be capable of handling the high rates expected in the \dune{} \gls{nd} environment.

The combination of results from all of this work builds the groundwork for the \AC{}, a novel, fully modular \lartpc{} concept, addressing the most important challenges of future neutrino detectors, in particular the \dune{} \gls{nd}.
High cathode voltages are prevented by splitting the detector into several small, self-contained \glspl{tpc} requiring only a moderate \SI{50}{\kilo\volt} cathode voltage.
A pixelated charge readout enables the true \gls{3d} tracking required to cope with the high event rates resulting from the high-intensity neutrino beam.
The \AL{} readout~\cite{arclight} minimises the dead volume in between the modules, resulting in a similar performance to a monolithic detector while containing the scintillation light within the modules, simplifying association to the correct ionisation signals.
The improvements I made led to \AC{} being the \lar{} component of the \dune{} \gls{nd} complex.