\section{Noise in Charge Readout Electronics}
\label{sec:studies_electronics}

\glsreset{snr}

For a heavy \gls{mip} with $\dv{E}{x} \approx \SI{2.1}{\mega\electronvolt\per\centi\metre}$ a \lartpc{} has a charge yield of $\sim{\SI{1}{\femto\coulomb\per\milli\metre}}$ as explained in Chapter~\ref{chap:lartpc}.
The readout electronics need to be able to reliably digitise this charge.
One of the biggest challenges to detect such low charges is the \gls{snr}.
This section gives a theoretical overview of various noise sources and mitigation techniques.
Noise can originate from a plethora of sources.
They can be divided into internal, originating inside the electronic components, and pick-up from external sources.

The most important internal source is the \emph{Johnson-Nyquist} noise.
It is generated by the intrinsic motion of the charge carriers at non-zero temperature and therefore often called thermal noise.
In statistical thermodynamics the energy of a system with one degree of freedom,
\begin{IEEEeqnarray}{rCl}
	E & = & \frac{k T}{2} \qc
\end{IEEEeqnarray}
is proportional to its temperature $T$ by the Boltzmann constant $k$.
The stored energy in a capacitor is given by
\begin{IEEEeqnarray}{rCl}
	E & = & \frac{C V ^ 2}{2} \qc
\end{IEEEeqnarray}
where $C$ is the capacitance of and $V$ the voltage across the capacitor.
Therefore, the voltage generated by the thermal noise inside an isolated ideal capacitor is
\begin{IEEEeqnarray}{rCl}
	V & = & \sqrt{\frac{k T}{C}} \qq*{.}
\end{IEEEeqnarray}
Combining this with the charge in the capacitor,
\begin{IEEEeqnarray}{rCl}
	\label{eq:electronics_thermal-noise}
	Q & = & C V = \sqrt{k T C} \qc
\end{IEEEeqnarray}
yields the equivalent noise charge due to the capacitor's temperature.~\cite{noise}

Equation~\eqref{eq:electronics_thermal-noise} has two important consequences for charge detectors: Noise scales with temperature and detector capacitance.
The temperature dependence is one of the main reasons to operate all analogue electronics at cryogenic temperatures.
Noise levels on pixels are significantly lower compared to wires due to the much smaller capacitance.

Another internal noise source are resonances in the signal path that can start to oscillate.
Resonances can occur from the combination of the impedance of electronic components such as cables and input impedances.
The main culprits are usually parasitic impedances not taken into account during the design of the circuit.
The resulting oscillations are superimposed on the signal.

An example of such a resonance is the behaviour of the cryogenic LARASIC preamplifiers used for \AT{}, described in Section~\ref{sec:studies_at-ro}.
They include a user-configurable shaping filter.
With its change the input capacitance of the amplifier changes as well.
Some configurations can form resonances with the circuit at the input.
Most passive electronic components change their values more or less significantly with temperature.
Therefore, the resonance behaviour of the detector circuit is different at room temperature and in \lar{}.
Additionally, every deviation from the final setup potentially changes parasitic impedances.
As a result, it is quite challenging to debug such resonances in the signal path.

\begin{figure}[tbp]
	\centering
	\includegraphics[width=\textwidth]{electronics/DiffSignaling}
	\caption[Differential signalling]{%
		Noise reduction using differential signalling.~\cite{diff_signal}
	}
	\label{fig:electronics_diff-signal}
\end{figure}

External sources can induce voltages on the signal path via variable magnetic fields, as predicted by Faraday's law.
Particularly prone to this are ground loops, any closed circuit supposed to be entirely at ground potential.
If the resistance at one place of the loop is high enough, the induction results in a voltage difference along the loop.
If the same part of the loop is used as reference of a signal carrying connection, the difference in the ground reference between signal source and sink will affect the signal.

There are several possibilities to make a circuit more resilient to external noise sources.
An obvious one is shielding all sensitive parts from external magnetic fields using a Faraday cage.
Implementing this effectively is extremely complicated and often not practical for small experiments.
Another approach is hardening the signal path itself by using current-coupled and/or symmetric signals.
Current-coupled signals are much less sensitive to induced voltages, as long as the voltages are small enough and do not result in significant current across parasitic impedances.
An example is \gls{nim} logic.

In conventional single-ended signalling the signal is measured as the voltage or current difference between a signal conductor and a ground common to signal source and sink.
Using a common ground as signal return path can have several undesired effects.
The signal conductor is usually enclosed in a ground shield.
If the shield is connected on both sides, a ground loop can result in combination with a shared power supply ground, for instance.
Ground loops can pick up noise through induction if the resistance along the loop is high enough.
A second way to couple noise into a single-ended system is by shifting the potential on the common ground away from the reference voltage or current, for instance due to high currents flowing through a lossy ground connection.
The signal will be distorted because it is always measured against the common ground.
In symmetric or differential signalling the signal is not measured between a signal conductor and ground, but instead between two signal conductors.
This works by putting an inverted (symmetric) waveform of the signal on a second conductor.
The signal is recovered by forming the difference between the two signal conductors.
As a result, the signal sink needs not be connected to the same ground as the signal source because the signal is independent of ground.
Ground loops in the signal path can thus be avoided.
Furthermore, the effect of noise pick-up on the signal lines is drastically reduced.
Inductive noise pick-up is equal on both signal conductors due to the completely symmetric signal path, as opposed to single-ended signals where the signal path is not symmetric.
In the signal sink the difference between the two symmetric signal conductors is formed and everything that is present on both of them, such as the inductively picked up noise, cancels out.

Disentangling the three different sources of noise (thermal noise, resonances, and external pick-up) is not easy.
Hints can often be found in the spectrum of the noise.
Thermal noise is equal and uncorrelated over the full frequency spectrum.
Resonances usually occur at specific frequencies and thus produce regular patterns, such as a sine, at the resonance frequency.
External sources are more difficult to identify.
If the source produces \gls{em} fields at known frequencies (e.g.\ harmonics of a switched power supply) the noise spectrum can be scanned for them.
Debugging is much more complex if the source is unknown.


\section{\AT{} Charge Readout Chain}
\label{sec:studies_at-ro}

\glsreset{mg}

\begin{figure}[tbp]
	\centering
	\includegraphics[width=\textwidth]{electronics/ReadoutChain_AT}
	\caption[\AT{} charge readout chain]{%
		Scheme of the \AT{} charge readout chain.
		From left to right: preamplifier power and control \acrshort{nim} module, LARASIC preamplifiers on wire readout plane, buffer amplifiers, CAEN V1724 digitisers.
		The dashed rectangle denotes the cold part of the chain.~\cite{AT_larasic}
	}
	\label{fig:viper_readoutChain_AT}
\end{figure}

Contemporary electronics schemes are introduced by looking at the existing readout chain at \gls{help}.
It was originally designed for the \AT{} experiment and a more detailed description can be found in~\cite{AT_larasic}.
I successfully upgraded the chain to partial differential signalling, significantly improving the \gls{snr}.

The charge collected by the readout plane is amplified by LARASIC4*~\cite{larasic} cryogenic charge amplifiers developed by \gls{bnl} for the \uboone{}~\cite{uboone}.
A performance characterisation of these \glspl{asic} can be found in~\cite{AT_larasic}.
Their main features include:
\begin{itemize}
	\item \num{16} channels per \gls{asic}
	\item low noise charge amplifier incorporating high-order filters
	\item per channel programmable gain of \SIlist[list-final-separator = {, or }]{4.7; 7.8; 14; 25}{\milli\volt\per\femto\coulomb}
	\item per channel programmable filter peaking time of \SIlist[list-final-separator = {, or }]{0.5; 1.0; 2.0; 3.0}{\micro\second}
	\item built-in test capacitance connected to dedicated external test pulse input for calibration
	\item power dissipation $< \SI{10}{\milli\watt}$ per channel
\end{itemize}

The cryogenic preamplifiers are mounted as close as possible to the readout in order to minimise noise pick-up on these very sensitive lines.
LARASICs can be programmed to the different aforementioned configurations via a \gls{spi}.
For this purpose they are connected to a bespoke \gls{nim} module housing an \gls{spi} controller, a test pulse generator, and multiple low-noise voltage regulators providing power to the LARASICs.
A standard PC controls the \gls{nim} module via \gls{usb}.
The output of the preamplifiers is fed to buffer amplifiers mounted on top of the cryostat signal feedthrough by means of flexible Kapton ribbon cables.
The buffers operate at room temperature, have a unity gain, and match the output impedance of the LARASICs to the \SI{50}{\ohm} input impedance of the downstream digitisers.
From the buffers the signals are routed via \SI{50}{\ohm} coaxial lines to \emph{CAEN V1724} digitisers~\cite{caen_adc} mounted in a \gls{vme} crate.
For debugging purposes the output of the buffers can be routed to an oscilloscope via a coaxial T-piece.
Finally, the digital data is read out from the \gls{vme} crate via a fibre-optic link by a standard PC.
Figure~\ref{fig:viper_readoutChain_AT} depicts the entire readout chain.
The complete analogue signal path from the pixel plane to the \gls{vme} digitisers is single-ended and thus prone to ground loops and all associated noise problems.

\begin{figure}[tbp]
	\centering
	\includegraphics[width=\textwidth]{noise/event500_rawUnfilteredPixel}\\
	\includegraphics[width=\textwidth]{noise/event500_rawUnfilteredROI}
	\caption[Event from first pixel demonstrator measurement campaign]{%
		Event from the first pixel demonstrator measurement campaign.
		The top plot shows pixel data while the bottom plot shows \acrshort{roi} data.
		Note that the colour scale does not represent the full available dynamic range.
	}
	\label{fig:electronics_event-run1}
\end{figure}

\begin{figure}[tbp]
	\centering
	\includegraphics[page=4, width=\textwidth]{noise/noise_run1} \\
	\includegraphics[page=1, width=\textwidth]{noise/noise_run1}
	\caption[Noise distributions from first pixel demonstrator measurement campaign]{%
		Noise amplitude distributions of pixel (top) and \acrshort{roi} (bottom) channels from the first pixel demonstrator measurement campaign.
		\num{5000} events from a \SI{5}{\hertz} random trigger, with \num{1000} \SI{410}{\nano\second} samples each, were combined.
	}
	\label{fig:electronics_noise-run1}
\end{figure}

\begin{figure}[tbp]
	\centering
	\includegraphics[width=\textwidth]{viper/event967_rawUnfilteredPixel}\\
	\includegraphics[width=\textwidth]{viper/event967_rawUnfilteredROI}
	\caption[Event from second pixel demonstrator measurement campaign]{%
		Event from the second pixel demonstrator measurement campaign, after implementing hardware noise mitigation measures.
		The top plot shows pixel data while the bottom plot shows \acrshort{roi} data.
		Note that the colour scale does not represent the full available dynamic range.
	}
	\label{fig:electronics_event-run2}
\end{figure}

\begin{figure}[tbp]
	\centering
	\includegraphics[page=4, width=\textwidth]{noise/noise_run2} \\
	\includegraphics[page=1, width=\textwidth]{noise/noise_run2}
	\caption[Noise distributions from second pixel demonstrator measurement campaign]{%
		Noise amplitude distributions of pixel (top) and \acrshort{roi} (bottom) channels from the second pixel demonstrator measurement campaign, after implementing hardware noise mitigation measures.
		\num{2000} events from a \SI{5}{\hertz} random trigger, with \num{2000} \SI{210}{\nano\second} samples each, were combined.
	}
	\label{fig:electronics_noise-run2}
\end{figure}

During the first pixelated readout measurement campaign (see Section~\ref{sec:ac_viper}) it became apparent that the data was significantly impaired by noise.
As can be seen in Figure~\ref{fig:electronics_event-run1}, the noise amplitude is similar over multiple channels.
This implies a common-mode component that cannot originate from inductive pick-up.
Instead, the noise is likely generated by self-oscillating parts of the signal path due to ground loops and parasitic impedances.
For the second measurement campaign different steps were taken to mitigate this behaviour through modifications to detector location, power supply, signal path, and intrinsic capacitance.

A correlation between noise levels and the running state of the air conditioning system in the utility room next to the lab was found.
Therefore, the experimental setup was moved away from the wall facing the utility room.

A decoupled clean power grid was built in the lab.
A \gls{mg} separates the lab grid mechanically from the building power supply.
Thus, any noise present on the latter is prevented from entering the experimental setup.
Furthermore, this decouples the lab grid entirely from the building ground preventing ground loops via electric mains.

The signal path from the impedance-matching buffer amplifiers to the digitisers---i.e.\ the warm signal path---was changed from single-ended to differential signalling.
This was achieved by replacing the buffer amplifiers by single-ended-to-differential amplifiers, and inserting another stage upstream of the digitisers to change the signal back to \SI{50}{\ohm} single-ended, matching the input of the digitisers.
Like this noise pick-up outside the cryostat could be reduced as well as sensitivity to ground loops between the detector and the \gls{daq} rack.
The design for the two buffer stages was kindly provided by the \lariat{} collaboration (see Section~\ref{sec:ac_pixlar} and~\cite{lariat}).

A source of noise was identified in the layout of the pixel readout plane.
It was found that due to several ground planes and long traces in the \gls{pcb} parasitic capacitances were very high.
Pixel channels are particularly affected due to the increased total trace lengths from connecting multiple pixels to the same \gls{daq} channel.
This is problematic because the input is shorted to ground for high enough frequencies (determined by $RC$), creating a ground loop.
Through the capacitive coupling to ground the system can start to oscillate.
One evidence for this is that the noise is equal over multiple channels, implying a common-mode component.
More specifically, the noise is equal for two respective groups of channels (see Figure~\ref{fig:electronics_event-run1}).
Investigating this, I found out that these groups correspond to channels of roughly equal parasitic capacitance: \SI{150 +- 5}{\pico\farad} and \SI{95 +- 5}{\pico\farad}.
The noise amplitude is higher on channels with higher capacitance (see Figure~\ref{fig:electronics_noise-run1}).
To solve this problem the \gls{pcb} design was optimised by removing unnecessary ground planes, routing signal tracks outside necessary ground planes, and increasing the thickness of the \gls{pcb}.
Pixel capacitance could be improved to \SI{65 +- 5}{\pico\farad} for all channels.
\gls{roi} capacitance improved only slightly from \SI{25 +- 10}{\pico\farad} to \SI{20 +- 10}{\pico\farad}, which confirms the hypothesis that the long traces due to pixel multiplexing were the culprits.
The reason for the higher spread of the \gls{roi} capacitances is the larger difference in trace length between the different \glspl{roi}.
For the sake of completeness it should be noted here that the old \gls{pcb} was not populated for the capacitance measurements while the new one was populated as described in Section~\ref{sec:ac_viper}.
However, the installed capacitors are either not connected to ground or in series with a \SI{10}{\mega\ohm} resistor.
Therefore, their influence on the measurements is negligible.

As can be seen from Figures~\ref{fig:electronics_event-run1} and~\ref{fig:electronics_event-run2}, there was a significant decrease in noise after applying all of the above improvements to the readout chain.
This can also be seen from Figures~\ref{fig:electronics_noise-run1} and~\ref{fig:electronics_noise-run2}, depicting the noise amplitude distributions of the two measurement campaigns.
The data for the noise distributions (\num{5000} events in the first and \num{2000} events in the second campaign) was taken employing a \SI{5}{\hertz} random trigger.
A more detailed assessment of the noise after the implementation of the described noise mitigation measures can be found in Section~\ref{sec:ac_viper}.


\section{Improved Cold Electronics for Pixelated Charge Readouts}
\label{sec:studies_pixel-electronics}

\glsreset{larpix}

This section describes the challenges met by electronics for pixelated \lartpc{}s and possible solutions.
I evaluated the cryogenic \glspl{adc} for the \dune{} \gls{fd}, developed by \gls{bnl}, and found that they are unsuitable for a pixelated \gls{nd}.
The neutrino group at \gls{lbnl} is developing bespoke pixel electronics for the \gls{nd}, the \larpix{}.
Based on my experience I advised the \gls{lbnl} group on the testing of their new readout electronics.

As mentioned in Section~\ref{sec:lartpc_electronics}, cold digitisation can improve noise because of both shorter analogue signal paths and reduced thermal noise of the electronics.
Furthermore, it enables data multiplexing on high-speed digital links, reducing the number of needed signal cables and cryostat feedthroughs.
However, designing reliable electronics at cryogenic temperatures is not an easy task.
\glspl{adc} require very stable reference voltages for proper analogue-to-digital conversion, making them susceptible to voltage fluctuations.
A further important aspect is power dissipation.
All power dissipated by cryogenic electronics needs to be compensated for in order to prevent the \lar{} from boiling.
This is particularly problematic for a pixelated readout that requires a much higher number of readout channels than a wire readout (see Section~\ref{sec:studies_pixel-ro}).
Another problem arises from the fact that digital electronics in general require clocks with sharp edges for proper timing, usually realised as a square wave.
According to Fourier analysis a square wave produces a high level of harmonics.
This is particularly problematic in the case of readout wires that can act as antennae and pick up these clock signals.

\begin{figure}[tbp]
	\centering
	\includegraphics[width=\textwidth]{bnl/bnl_adc_lin}
	\caption[Linearity measurement of \glsentryshort{bnl} cryogenic \glsentryshort{adc} \glsentryshortpl{asic}]{%
		Linearity measurement of the \acrshort{bnl} cryogenic \acrshort{adc} \acrshortpl{asic} with input voltage on the x-axis and \acrshort{adc} value (code) on the y-axis.
		Colour represents the number of measurements.
		The measurements were performed in liquid nitrogen.
	}
	\label{fig:bnl_adc_lin}
\end{figure}

\gls{bnl} is developing cold charge readout electronics for the \dune{} \gls{fd}~\cite{protodune-sp}.
The plan is to accompany the cryogenic LARASIC charge preamplifiers by cryogenic \glspl{adc}.
They have \num{16} inputs, each capable of digitising the \gls{tpc} signals at \SI{2}{\mega{}S\per\second} and \SI{12}{\bit} with input characteristics optimised for the LARASIC output.
A more detailed description is given in~\cite{bnl_adc}.

In the course of this work, the cryogenic \gls{adc} \glspl{asic} developed by \gls{bnl} were evaluated to be used in the \gls{nd} as well.
I joined the team at \gls{bnl} in cold tests of the devices.
One of the results of these tests is presented here to illustrate the difficulties of cryogenic \glspl{adc}.
As a disclaimer, it should be noted that this is by no means the status of the \glspl{adc} at the time of writing.
The described tests were performed in autumn 2016 at \gls{bnl}.

An important characteristic of an \gls{adc} is linearity.
It describes the relation between the applied input voltage and the calculated digital number, the \emph{\gls{adc} code}, at the output.
In case of the \gls{bnl} \glspl{adc} this relation is expected to be strictly linear.
To test this a voltage ramp is applied to the input and the converted digital values are recorded.

\begin{table}[tbp]
	\centering
	\caption[\glsentryshort{larpix} specifications]{%
		\acrshort{larpix} specifications for a \SI{10}{\mega\hertz} external clock.~\cite{larpix_spec}
	}
	\label{tab:larpix_spec}
	\begin{tabu} to \textwidth {lSs}
		\toprule
		Specification &													{Value} &											{Unit} \\
		\midrule
		Number of analogue inputs (channels) &							32 &												\\
		Noise at \SI{88}{\kelvin} &										300 &												\elementarycharge \\
		Noise at \SI{300}{\kelvin} &									500 &												\elementarycharge \\
		Channel gain &													{\numlist[list-pair-separator = { or }]{4; 45}} &	\micro\volt\per\elementarycharge \\
		Time resolution &												2 &													\micro\second \\
		Analogue dynamic range &										\approx 1300 &										\milli\volt \\
		\acrshort{adc} resolution &										8 &													\bit \\
		Threshold range &												\numrange{0}{1.8} &									\volt \\
		Threshold resolution &											< 1 &												\milli\volt \\
		Channel linearity &												1 &													\percent \\
		Operating temperature range &									\numrange{80}{300} &								\kelvin \\
		Event memory depth &											2048 &												\\
		Nominal output signalling level (\acrshort{cmos}) &				3.3 &												\volt \\
		Digital data rate &												5 &													\mega\bit\per\second \\
		Event readout time &											\approx 11 &										\micro\second \\
		Power dissipation per channel at \SI{1}{\hertz} event rate &	\sim 100 &											\micro\watt \\
		\bottomrule
	\end{tabu}
\end{table}

A typical measurement is shown in Figure~\ref{fig:bnl_adc_lin}.
The expected shape is one straight diagonal line from the bottom left to the top right, i.e.\ a linear relationship between input voltage and \gls{adc} value.
Two particular deviations from this are visible: gaps accompanied by horizontal lines and a wobbly response around zero.
Upon close inspection it can be seen that the gaps have the same voltage range as the horizontal lines.
The meaning of this is that the \gls{adc} output is \emph{stuck} at the same value for the corresponding input voltage range.
Both effects result in a non-linear detector response to detected charge and thus energy deposition.
While some non-linearities can be compensated in offline data analysis, this is not possible for the sticking \gls{adc} values because they correspond to a range of input voltages.
This impairs the energy resolution of the detector.

The cause for the non-linearities is rooted in the electronic design of the \gls{asic}.
It was not fully understood at the time of these tests.
Therefore, an explanation is out of the scope of this work and not given here.
The measurements are shown to illustrate the difficulties of designing a reliable cryogenic \gls{adc}.

Leaving aside the non-linear response, the \gls{bnl} \glspl{adc} are not suitable for use in conjunction with a pixelated \lartpc{} charge readout.
Being designed for wire readouts, no strong focus was laid on power dissipation, which is $\approx \SI{5}{\milli\watt}$ per channel.
Combined with the one of the LARASIC (\SI{10}{\milli\watt})~\cite{larasic} a total of $\approx \SI{15}{\milli\watt}$ is dissipated.
A pixelated \dune{} \gls{nd} with a \SI{3}{\milli\metre} pitch and the dimensions given in Section~\ref{sec:dune-nd_ac-nd} will need $\gtrsim {\num{e7}}$ channels.
The resulting required cooling power would be $\gtrsim \SI{150}{\kilo\watt}$ for \SI{84}{\tonne} of \lar{}.
In comparison, \uboone{} has a total cooling power of $\approx \SI{20}{\kilo\watt}$ for \SI{170}{\tonne} of \lar{}~\cite{uboone}.

\begin{figure}[tbp]
	\centering
	\includegraphics[width=\textwidth]{larpix/schematic}
	\caption[\glsentryshort{larpix} block diagram]{%
		Conceptual block diagram of a \acrshort{larpix} channel.~\cite{dan_larpix_arcCM}
	}
	\label{fig:larpix_schematic}
\end{figure}

Due to their smaller geometric extent pixels have a much lower capacitance ($\approx \SI{4}{\pico\farad}$ for vias~\cite{larpix_spec}) than wires ($\approx \SI{200}{\pico\farad}$~\cite{protodune-sp}).
According to Equation~\eqref{eq:electronics_thermal-noise} this reduces the intrinsic noise present on a pixelated readout.
\larpix{}, being developed by \gls{lbnl}~\cite{larpix, larpix_spec}, exploits this fact to significantly reduce the complexity of the cold electronics.
Two key points distinguish them from the \gls{bnl} design for the wire-equipped \gls{fd}.
The complex shaping preamplifier required by wires for noise filtering can be replaced by a simple charge integrator.
Additionally, the low noise levels allow for a smart zero suppression scheme; charge arriving at the \larpix{} is only digitised if it is above a predefined threshold.
This reduces the duty cycle and thus power dissipation of the \gls{adc}.
If noise levels are well below the set threshold, power dissipation becomes primarily a function of charge flux rate in the detector.

In addition, the digital circuitry of \larpix{} operates at lower frequencies than the \gls{bnl} design.
For an \gls{ac} of frequency $f$, the resistance presented by a conductor is not simply given by its Ohmic resistance.
There is an additional component proportional to $\sqrt{f}$ caused by the \emph{skin effect}~\cite{horowitzHill}.
\gls{hf} currents do not flow in the bulk of the conductor but only in a finite layer (skin) at its surface.
Therefore, the conductivity is no longer proportional to the cross-section area but rather the circumference of the conductor.
The result of the skin effect is more power dissipation at higher frequencies for the same conductor geometry.
By operating at lower frequencies the power dissipation of \larpix{} can be lowered further.
The cost is a decrease in data transmission rates.

With its power dissipation dependent on the charge flux and the lowered data transmission rate \larpix{} is susceptible to high event rates.
The same goes for noise levels due to the self-triggered digitisation.
For the successful operation of \larpix{} it is of paramount importance to keep event rates and noise levels low.
The latter can be achieved by minimising detector capacitance.
To lower the susceptibility to high event rates the \dune{} \gls{nd} design orientates the \gls{tpc} drift direction perpendicular to beam direction.
This reduces the amount of charge per event arriving simultaneously at the readout.
Furthermore, \larpix{} is equipped with a \gls{fifo} buffer capable of holding \num{2048} charge pulses to cope with short peaks in event rate.

\begin{figure}[tbp]
	\centering
	\includegraphics[width=\textwidth]{larpix/daisy}
	\caption[\glsentryshort{larpix} daisy chain]{%
		\acrshort{larpix} daisy chain configuration.~\cite{larpix_spec}
	}
	\label{fig:larpix_daisy}
\end{figure}

To accommodate the elevated channel number of a pixelated readout the first \larpix{} prototype chip has \num{32} inputs.
Its resolution in time and charge are \SI{2}{\micro\second} and \SI{8}{\bit}, respectively.
While currently inferior to the \gls{bnl} design, these specifications are planned to be improved in the next design iteration.
The first \larpix{} version aims to demonstrate two critical aspects~\cite{dan_larpix_arcCM}:
\begin{enumerate}
	\item Low noise and low power dissipation (see Table~\ref{tab:larpix_spec})
	\item \gls{mip} track detection capability in a test \gls{tpc}
\end{enumerate}
Another goal is to assess the optimal size of the \gls{fifo} event buffer~\cite{danLarpix}.
The most important \larpix{} specifications are given in Table~\ref{tab:larpix_spec}.

Figure~\ref{fig:larpix_schematic} shows the block diagram of a single \larpix{} channel.
The incoming charge is converted to a voltage by a \gls{csa} via integration on the feedback capacitance $C$.
To minimise power dissipation the output of the \gls{csa} is only digitised if it is above a digitally configurable threshold.
This is realised by discriminating the signal against a \gls{dac} output.
If the signal is above threshold, the discriminator triggers a digitisation and then a reset of the \gls{csa}.
Each channel can be connected individually to an analogue monitor bus shared between multiple \larpix{} chips (see Figure~\ref{fig:larpix_daisy}).
The \larpix{} controller outside the cryostat can probe the analogue monitor bus to set the thresholds correctly (see the datasheet~\cite{larpix_spec} for a detailed procedure).

Digitisation is performed by a standard \gls{sar} digitiser.
It works as follows~\cite{horowitzHill}.
The input signal is compared to the output of a \gls{dac} controlled by a register.
At the start of the conversion all bits in the register are set to \num{0}.
Then, the most significant bit is set to \num{1}.
If the \gls{dac} output is above the analogue input, the bit is set back to \num{0}.
Otherwise, it is kept at \num{1}.
The procedure is successively repeated for all bits.
After the least significant bit has been set, the conversion is complete and the value of the register is output to the digital event buffer (\gls{fifo}).
The whole conversion requires the same number of clock cycles as the number of \gls{dac} (and therefore \gls{adc}) bits.

\begin{figure}[tbp]
	\centering
	\includegraphics[width=\textwidth]{larpix/example_track}
	\caption[\glsentryshort{larpix} prototype event]{%
		Cosmic muon track recorded with the first \acrshort{larpix} prototype.
		The left projection is perpendicular to the pixel plane while the right one is almost parallel to it.
		Note that only red pixels are instrumented.~\cite{danLarpix}
	}
	\label{fig:larpix_track}
\end{figure}

\larpix{} uses a \gls{uart} connected by \gls{spi} for communication.
This allows to daisy chain up to \num{256} \larpix{} chips as depicted in Figure~\ref{fig:larpix_daisy}.
The daisy chain is connected to a \larpix{} controller outside the cryostat, which can be a \gls{fpga} or any other digital controller capable of handling the expected data rate.
Chips are identified by means of a hard-coded \gls{id}.
During data taking each \larpix{} chip constantly transmits the events present in its \gls{fifo} buffer to the controller via \gls{spi}.
One data packet or event consists of \SI{54}{\bit} of data, containing a single \gls{adc} value, chip and channel \gls{id}, and a timestamp.
Configuration of the \larpix{} chips (e.g.\ threshold setting) happens via the same \gls{spi} line.

\larpix{} does not have its own oscillator.
All timing signals are derived from an external clock signal supplied by the controller and shared between multiple chips.
In particular, this clock defines the time resolution of the digitisation and the \gls{spi} data rate.
Furthermore, the event timestamp generated inside \larpix{} is derived from this clock.
The specifications in Table~\ref{tab:larpix_spec} are given for a \SI{10}{\mega\hertz}, clock resulting in a time resolution of \SI{2}{\micro\second}.
This is the nominal configuration for the first prototype.
Later chip designs will be optimised to allow for a better time resolution.

Each \larpix{} chip has an integrated test pulse generator.
It consists of a \gls{dac} that can be connected to one or more inputs via a coupling capacitor.
Charge can be injected into the selected inputs by switching the level of the \gls{dac}.
This allows to characterise and/or debug the charge readout chain in a similar fashion to the LARASIC preamplifiers described in Section~\ref{sec:studies_at-ro}.

One design goal of \larpix{} is to reach a power dissipation of \SI{100}{\micro\watt} per channel.
Using the same numbers as above this results in a total power dissipation of $\sim \SI{1}{\kilo\watt}$ for a pixelated \dune{} \gls{nd}.
Most of the supply voltages of \larpix{} can be reduced from their nominal value.
This allows to further decrease power dissipation in addition to the smart zero suppression and reduced clock frequency.
Attention has to be paid to adjust the \gls{spi} signalling levels accordingly.
More information on this can be found in the datasheet~\cite{larpix_spec}.

The first \larpix{} prototype was successfully tested at \gls{lbnl} in a shorter version of the pixel demonstrator \gls{tpc} described in Section~\ref{sec:ac_viper}.
In particular, the noise and power dissipation levels given in Table~\ref{tab:larpix_spec} were reached~\cite{dan_larpix_duneCM}.
Figure~\ref{fig:larpix_track} shows a recorded cosmic muon track.
Note that only the red pixels are instrumented, explaining the segmented track.

\section{Charge Readout Electronics Summary}
\label{sec:studies_electronics-summary}

Pixelated \lartpc{}s place high demands on the charge readout electronics.
The very high number of readout channels required makes digitisation outside of the cryostat impractical due to the resulting number of cable feedthroughs.
Cold digitisation inside the cryostat reduces the number of cables by channel aggregation on digital high-speed links.
However, this worsens the problem of heat dissipation inside the \lar{}.
I evaluated the cold digitisers developed for the charge readout wire planes of the \dune{} \gls{fd}, but found them to be unsuitable for a pixelated \lartpc{} due to their high power dissipation.
\larpix{} is a bespoke cold digitiser for pixelated \lartpc{}s.
It is currently under development at \gls{lbnl} and designed to meet the stringent heat dissipation requirements by means of a smart zero suppression.