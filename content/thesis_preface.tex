\renewcommand{\Chapter}{{Preface}}
\chapter*{\Chapter}
\chaptermark{\Chapter}

This thesis studies most of the relevant challenges for \lartpc{} neutrino detectors in future high-multiplicity environments alongside potential solutions, namely the dielectric strength of \lar{}, new charge and light readout methods, as well as the required next-generation charge readout electronics.
Chapter~\ref{chap:introduction} sets the stage and motivates my work, it is a combination of various sources.

The theoretical background of neutrino detection and oscillation is elucidated in Chapter~\ref{chap:nu-detection}.
It is started with a short historical introduction loosely based on Giunti and Kim~\cite{giunti} who also provide a very detailed overview of neutrino physics.
Details on the detectors used in the historical experiments are taken from Grupen and Schwartz~\cite{grupen}, as is the section on final state detection.
The theory of neutrino oscillations is inspired by Schmitz' book on neutrino physics~\cite{schmitz}.

Chapter~\ref{chap:lartpc} introduces the \lartpc{} detector with all its subsystems and peculiarities.
It is based on the book on ``Noble Gas Detectors'' by E.\ Aprile et al.~\cite{NobleGasDetectors} and the \gls{help} Master thesis of M.\ Schenk~\cite{michu}.

Various studies of the technologies required by \AC{} are presented in Chapter~\ref{chap:studies}.
Most of this is my work.
I made crucial contributions to the setup of the electric breakdown measurements and played a leading role in data analysis and writing of the paper presenting the results~\cite{breakdown_16}, of which I am corresponding author.
These studies resulted in a second paper~\cite{latex} on a method to mitigate breakdowns, which I co-authored.
With the \gls{hv} issues addressed I started investigating new charge readout technologies.
Two technologies are presented: a replacement for wires using copper traces on a thin Kapton layer and a pixelated readout.
Both ideas are not new but they have never been used in \lar{} before.
I built, commissioned and operated the test setup including a small \gls{tpc} for the copper on Kapton readout.
The section on the pixelated readout introduces the theory of the applied analogue multiplexing scheme based on~\cite{maplesyrup}.
Also described is a composite effort I lead to reduce the noise present in the setup used to test the pixelated readout.
Crucial input on the electronics modifications was kindly provided by D.\ Shooltz from the \lariat{} collaboration.
Details on the test setup and results are presented in Chapter~\ref{chap:ac}.
At \gls{bnl}, NY, USA I tested new cold charge readout electronics.
Based on the knowledge gained from these tests, I advised the neutrino group at \gls{lbnl}, CA, USA on the testing of their new bespoke pixel electronics, \larpix{}.
Also presented in this chapter are cold \gls{sipm} tests performed with the pixel demonstrator described in Chapter~\ref{chap:ac}.
They enabled the development of \AL{}~\cite{arclight} by the \gls{help} \lar{} group, which I helped testing and characterising.

Chapter~\ref{chap:ac} presents the novel \AC{} \lartpc{} concept developed at \gls{help}.
I designed and constructed a bigger \gls{tpc}, the pixel demonstrator, to test the pixelated charge readout.
I lead its commissioning and operation.
For the analysis of the recorded cosmic muon tracks I wrote a reconstruction framework from scratch.
Y.-T.\ Tsai and T.\ Usher from SLAC, CA, USA provided valuable input on the employed reconstruction algorithms.
Based on these findings, a scaled-up version of the pixelated readout was placed in \lariat{} with my relevant contributions to the pixel plane design and detector operation.
I have presented the pixel demonstration at several conferences (e.g.~\cite{pixel_proceedings}) and am corresponding author of a resulting paper~\cite{pixel_paper}.
All the aforementioned work went into the design of \AC{} described in the last section of this chapter.
The design is the work of the \AC{} collaboration and will be written up in an appropriate document in the near future.

The detailed implementation of \AC{} in the \dune{} \gls{nd} is described in Chapter~\ref{chap:dune-nd}.
Again, this is the work of the \AC{} collaboration based on the findings I present in this thesis.
To support our proposal of \AC{} for the \dune{} \gls{nd} we needed to prove its ability to cope with the high rates expected.
I provide this proof in the last section of Chapter~\ref{chap:dune-nd} using a reconstruction simulation I wrote based on my previous findings on the performance of pixelated \lartpc{}s.
C.\ Marshall from \gls{lbnl}, CA, USA kindly provided guidance and the raw simulated neutrino events.

The thesis is wrapped up in Chapter~\ref{chap:conclusion}.
This is my work.