\renewcommand{\Chapter}{{Preface}}
\chapter*{\Chapter}
\chaptermark{\Chapter}

\todo[inline, color=red]{preface}

Chapter~\ref{chap:introduction} sets he stage, it is a combination of various sources.

The theoretical background of neutrino detection and oscillation is elucidated in Chapter~\ref{chap:nu-detection}.
It is started with a short historical introduction loosely based on Giunti and Kim~\cite{giunti} who also provide a very detailed overview of neutrino physics.
Details on the detectors used in the historical experiments are taken from Grupen and Schwartz~\cite{grupen} as is the section on final state detection.
The theory of neutrino oscillations is inspired by Schmitz' book on neutrino physics~\cite{schmitz}.

Chapter~\ref{chap:lartpc} introduces the \lartpc{} detector with all its subsystems and peculiarities.
It is based on the book on ``Noble Gas Detectors'' by Aprile et al.~\cite{NobleGasDetectors} and the \gls{help} Master thesis of M.\ Schenk~\cite{michu}.

Various studies of the technologies required by \AC{} are presented in Chapter~\ref{chap:studies}.
Most of this is my work.
I made crucial contributions to the setup of the electric breakdown measurements and played a leading role in data analysis and writing of the paper presenting the results~\cite{breakdown_16} of which I am corresponding author.
From these studies a second paper~\cite{latex} resulted on a method to mitigate breakdowns, which I co-authored.
With the \gls{hv} issues addressed, I started investigating new charge readout technologies.
Two technologies are presented: a replacement for wires using copper tracks on a thin Kapton layer and a pixelated readout.
Both ideas are not new but they have never been used in \lar{} before.
I built, commissioned and operated the test setup including a small \gls{tpc} for the copper on Kapton readout.
The section on the pixelated readout introduces the theory of the applied analogue multiplexing scheme based on~\cite{maplesyrup}.
Also described is a composite effort I lead to reduce the noise present in the setup used to test the pixelated readout.
Crucial input on the electronics modifications was kindly provided by D.\ Shooltz from the \lariat{} collaboration.
Details on the test setup and results are presented in Chapter~\ref{chap:ac}.
At \gls{bnl}, NY, USA I tested new cold charge readout electronics.
Based on the knowledge gained from these tests, I advised the neutrino group at \gls{lbnl} on the testing of their new bespoke pixel electronics, \larpix{}.
The presented cold \gls{sipm} tests were part of the pixel demonstrator described in Chapter~\ref{chap:ac}.
They enabled the development of \AL{} by the \gls{help} \lar{} group which I helped testing and characterising.

Chapter~\ref{chap:ac} presents the novel \AC{} \lartpc{} concept developed at \gls{help}.
A bigger \gls{tpc}, the pixel demonstrator, was built by the \gls{help} \lar{} group to test the pixelated charge readout.
I was heavily involved in the design and construction and lead the commissioning of the detector as well as the second measurement campaign.
For the analysis of the recorded cosmic muon tracks I wrote a reconstruction framework from scratch.
Y.-T.\ Tsai and T.\ Usher from SLAC, CA, USA provided valuable input on the employed reconstruction algorithms.
Based on these findings, a scaled-up version of the pixelated readout was placed in \lariat{}.
All the aforementioned work went into the design of \AC{} described in the last section of this chapter.
The design is the work of the \AC{} collaboration and will be written up in an appropriate document in the near future.

The detailed implementation of \AC{} in the \dune{} \gls{nd} is described in Chapter~\ref{chap:dune-nd}.
Again, this is the work of the \AC{} collaboration based on the findings presented in this thesis.
To support our proposal of \AC{} for the \dune{} \gls{nd} we needed to prove its ability to cope with the high rates expected.
I provide this proof in the last section of this chapter.
It is was supported by valuable input from C.\ Marshall from \gls{lbnl}, CA, USA who in particular kindly provided the raw simulated neutrino events used for this study.

The thesis is wrapped up in Chapter~\ref{chap:conclusion}.
This is my work.